\begin{uebsp}
\begin{Exercise}[label=ex:1.7]
Ein Würfel wird dreimal geworfen. Die Augenzahlen werden der Größe nach geordnet, die Zufallsvariable $X$ sei die mittlere (etwa $(2, 2, 5) \rightarrow 2$).\\
Bestimmen Sie die Verteilung von $X$.
\end{Exercise}
\begin{Answer}
    \begin{enumerate}[1.]
        \item Schritt: 4 Fälle betrachten:
                \begin{enumerate}[I)]
                    \item $y=X=z\;\Rightarrow\;$1 Permutation ($\dfrac{3!}{3!}$)
                    \item $y=X<z\;\Rightarrow\;$3 Permutationen ($\dfrac{3!}{2!}$)
                    \item $y<X=z\;\Rightarrow\;$3 Permutationen ($\dfrac{3!}{2!}$)
                    \item $y<X<z\;\Rightarrow\;$6 Permutationen ($\dfrac{3!}{1!}$)\\
                        z.B. kann $2<3<4$ durch folgende 6 Würfelkonstellationen/Permutationen entstehen:
                        $(2,3,4)$, $(2,4,3)$, $(3,2,4)$, $(3,4,2)$, $(4,2,3)$, $(4,3,2)$
                \end{enumerate}
        \item Schritt: Bestimmen der Wahrscheinlichkeit für X=k:
            \[X=1:\;1\cdot I+5\cdot II+0\cdot III+0\cdot IV=1\cdot1+5\cdot3+0\cdot3+0\cdot6=16\]
                \[\Rightarrow\mathbb{P}(X=1)=\frac{16}{216}\approx \;\;7.4\%\]
            \[X=2:\;1\cdot I+4\cdot II+1\cdot III+4\cdot IV=1\cdot1+4\cdot3+1\cdot3+4\cdot6=40\]
                \[\Rightarrow\mathbb{P}(X=2)=\frac{40}{216}\approx 18.5\%\]
            \[X=3:\;1\cdot I+3\cdot II+2\cdot III+6\cdot IV=1\cdot1+3\cdot3+2\cdot3+6\cdot6=52\]
                \[\Rightarrow\mathbb{P}(X=3)=\frac{52}{216}\approx 24.1\%\]
            \[X=4:\;1\cdot I+2\cdot II+3\cdot III+6\cdot IV=1\cdot1+2\cdot3+3\cdot3+6\cdot6=52\]
                \[\Rightarrow\mathbb{P}(X=4)=\frac{52}{216}\approx 24.1\%\]
            \[X=5:\;1\cdot I+1\cdot II+4\cdot III+4\cdot IV=1\cdot1+1\cdot3+4\cdot3+4\cdot6=40\]
                \[\Rightarrow\mathbb{P}(X=5)=\frac{40}{216}\approx 18.5\%\]
            \[X=6:\;1\cdot I+0\cdot II+5\cdot III+0\cdot IV=1\cdot1+0\cdot3+5\cdot3+0\cdot6=16\]
                \[\Rightarrow\mathbb{P}(X=6)=\frac{16}{216}\approx \;\;7.4\%\]
        \item Verteilungsfunktion $F_X(x)$:
            \begin{multicols}{2}

                \begin{tikzpicture}[scale=0.85]
                    {
    \begin{axis}[domain=0:8,
            axis x line=bottom, % no box around the plot, only x and y axis
            axis y line=left, % the * would suppress the arrow tips
            xlabel=Augenzahl,
            ylabel=Prozent,
            legend pos=north west,
            samples=50,
            height=6cm,
            width=10cm,
            clip=false]
            \addplot[blue] coordinates{(-1,0)(1,0)};
            \addlegendentry[align=left]{Vert.fkt. $f_X(x)$};%TODO: add a better label
            \addplot[blue,forget plot] coordinates{(1,7.4)(2,7.4)};
            \addplot[blue,forget plot] coordinates{(2,25.9)(3,25.9)};
            \addplot[blue,forget plot] coordinates{(3,50)(4,50)};
            \addplot[blue,forget plot] coordinates{(4,74.1)(5,74.1)};
            \addplot[blue,forget plot] coordinates{(5,92.6)(6,92.6)};
            \addplot[blue,forget plot] coordinates{(6,100)(8,100)};

            \draw[dotted] (axis cs:1,0) -- (axis cs:1,7.4);
            \draw[dotted] (axis cs:2,7.4) -- (axis cs:2,25.9);
            \draw[dotted] (axis cs:3,25.9) -- (axis cs:3,50);
            \draw[dotted] (axis cs:4,50) -- (axis cs:4,74.1);
            \draw[dotted] (axis cs:5,74.1) -- (axis cs:5,92.6);
            \draw[dotted] (axis cs:6,92.6) -- (axis cs:6,100);
            \addplot[discontinuityblue,forget plot] coordinates{(1,0)(2,7.4)(3,25.9)(4,50)(5,74.1)(6,92.6)};
            \addplot[continuityblue,forget plot] coordinates{(1,7.4)(2,25.9)(3,50)(4,74.1)(5,92.6)(6,100)};


            \addplot[red] coordinates{(-1,0)(1,0)};
            \addlegendentry[align=left]{Vert. diskret};%TODO: add better label
            \addplot[red,forget plot] coordinates{(1,7.4)(2,7.4)};
            \addplot[red,forget plot] coordinates{(2,18.5)(3,18.5)};
            \addplot[red,forget plot] coordinates{(3,24.1)(5,24.1)};
            \addplot[red,forget plot] coordinates{(5,18.5)(6,18.5)};
            \addplot[red,forget plot] coordinates{(6,7.4)(7,7.4)};
            \addplot[red,forget plot] coordinates{(7,0)(8,0)};

            \draw[dotted] (axis cs:1,0) -- (axis cs:1,7.4);
            \draw[dotted] (axis cs:2,7.4) -- (axis cs:2,18.5);
            \draw[dotted] (axis cs:3,18.5) -- (axis cs:3,24.1);
            \draw[dotted] (axis cs:5,24.1) -- (axis cs:5,18.5);
            \draw[dotted] (axis cs:6,18.5) -- (axis cs:6,7.4);
            \draw[dotted] (axis cs:7,7.4) -- (axis cs:7,0);

            \addplot[discontinuityred,forget plot] coordinates{(1,0)(2,7.4)(3,18.5)(5,24.1)(6,18.5)(7,7.4)};
            \addplot[continuityred,forget plot] coordinates{(1,7.4)(2,18.5)(3,24.1)(5,18.5)(6,7.4)(7,0)};
    \end{axis}
}

                \end{tikzpicture}

                Die rote Linie gibt die Verteilung diskret an, während die blaue Linie die Verteilungsfunktion $f_X(x)$ rechts darstellt.
            \columnbreak
            \[f_X(x) = \begin{cases} 
                        0 &\mbox{wenn } x < 1 \\
                        \frac{16}{216}\approx 7.4\% & \mbox{wenn } x < 2\\
                        \frac{56}{216}\approx 25.9\% & \mbox{wenn } x < 3\\
                        \frac{108}{216}\approx 50\% & \mbox{wenn } x < 4\\
                        \frac{160}{216}\approx 74.1\% & \mbox{wenn } x < 5\\
                        \frac{200}{216}\approx 92.6\% & \mbox{wenn } x < 6\\
                        \frac{216}{216}\approx 100\% & \mbox{wenn } x < 7 
                        \end{cases}\]
            \end{multicols}
    \end{enumerate}
\end{Answer}
\end{uebsp}

