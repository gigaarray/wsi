\begin{uebsp}
\begin{Exercise}[label=ex:1.2]
Von einer Krankheit sind 2\% der Bevölkerung betroffen. Ein Test gibt
bei einem Kranken mit Wahrscheinlichkeit 0.99 ein positives Ergebnis bei
einem Gesunden mit Wahrschinlichkeit 0.01.
\Question
Bestimmen Sie die Wahrscheinlichkeit, dass eine zufällig gewählte Person positiv getestet wird.
\Question
Bestimmen Sie die bedingte Wahrscheinlichkeit dafür, dass eine zufällig gewählte Person krank ist, wenn das Testergebnis positiv ist.
\end{Exercise}
\begin{Answer}
\begin{enumerate}[(a)]
    \item In diesem Fall gibt es 2 Möglichkeiten:
        \begin{enumerate}[i)]
            \item Person ist krank und der Test ist positiv ($A\cap B$)
            \item Person ist gesund und der Test ist positiv ($A^c\cap B)$
        \end{enumerate}
        \begin{uebsp_theory}
        Mit dem \reference{Satz der vollständigen Wahrscheinlichkeit}{Satz}{satz:vollstaendige_wahrscheinlichkeit}
        \[\mathbb P(A)=\sum_i\mathbb P(B_i)\mathbb P(A|B_i).\]
        folgt:
        \end{uebsp_theory}
        \index{vollständige Wahrscheinlichkeit, Satz!Beispiel}
        \[\mathbb{P}(B)=\mathbb{P}(B|A)\cdot\mathbb{P}(A)+\mathbb{P}(B|A^c)\cdot\mathbb{P}(A^c)\]
        \[\mathbb{P}(B)=\frac{99}{100}\cdot\frac{2}{100}+\frac{1}{100}\cdot\frac{98}{100}=\frac{198}{10k}+\frac{98}{10k}=\frac{296}{10k}\approx3\%\]
    \item $\mathbb{P}(A|B)$ ist gesucht: 
        \begin{uebsp_theory}
            Mit der \reference{Bedingten Wahrscheinlichkeit}{Definition}{def:bedingte_wahrscheinlichkeit} und dem \reference{Satz von Bayes}{Satz}{satz:bayes}
            \[\mathbb P(B|A)=\frac{\mathbb P(B)\mathbb P(A|B)}{\mathbb P(A)}\]
            folgt:
            \index{bedingte Wahrscheinlichkeit!Beispiel}
        \end{uebsp_theory}
        \[\mathbb{P}(A|B)=\frac{\mathbb{P}(B|A)\mathbb{P}(A)}{\mathbb{P}(B)}=\frac{\frac{99}{100}\cdot\frac{2}{100}}{\frac{296}{10k}}=\frac{99\cdot 2}{10k}\cdot\frac{10k}{296}\approx 0.669\approx 67\%\]

\end{enumerate}
\end{Answer}
\end{uebsp}
