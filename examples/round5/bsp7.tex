% This file was converted to LaTeX by Writer2LaTeX ver. 1.0.2
% see http://writer2latex.sourceforge.net for more info
%\documentclass[a4paper]{article}
%\usepackage[ascii]{inputenc}
%\usepackage[T1]{fontenc}
%\usepackage[english,ngerman]{babel}
%\usepackage{amsmath}
%\usepackage{amssymb,amsfonts,textcomp}
%\usepackage{color}
%\usepackage{array}
%\usepackage{hhline}
%\usepackage{hyperref}
%\hypersetup{pdftex, colorlinks=true, linkcolor=blue, citecolor=blue, filecolor=blue, urlcolor=blue, pdftitle=, pdfauthor=, pdfsubject=, pdfkeywords=}
% Text styles
%\newcommand\textstyleAbsatzStandardschriftart[1]{#1}
% Page layout (geometry)
%\setlength\voffset{-1in}
%\setlength\hoffset{-1in}
%\setlength\topmargin{2cm}
%\setlength\oddsidemargin{2cm}
%\setlength\textheight{23.698002cm}
%\setlength\textwidth{17.001cm}
%\setlength\footskip{0.751cm}
%%\setlength\headheight{1.251cm}
%\setlength\headsep{0cm}
% Footnote rule
%\setlength{\skip\footins}{0.119cm}
%\renewcommand\footnoterule{\vspace*{-0.018cm}\setlength\leftskip{0pt}\setlength\rightskip{0pt plus 1fil}\noindent\textcolor{black}{\rule{0.25\columnwidth}{0.018cm}}\vspace*{0.101cm}}
% Pages styles
%\makeatletter
%\newcommand\ps@MP{
%  \renewcommand\@oddhead{\"Ubung 5\hfill \hfill 27.11.2013}
%  \renewcommand\@evenhead{\@oddhead}
%  \renewcommand\@oddfoot{Wahrscheinlichkeitsrechnung\hfill Markus Kessler\hfill Seite %\textstyleAbsatzStandardschriftart{\textbf{\thepage{}}} von %\textstyleAbsatzStandardschriftart{\textbf{?}} \textstyleAbsatzStandardschriftart{und %stochastische Prozesse}}
%  \renewcommand\@evenfoot{\@oddfoot}
%  \renewcommand\thepage{\arabic{page}}
%}
%\newcommand\ps@Standard{
%  \renewcommand\@oddhead{}
%  \renewcommand\@evenhead{}
%  \renewcommand\@oddfoot{}
%  \renewcommand\@evenfoot{}
%  \renewcommand\thepage{\arabic{page}}
%}
%\makeatother
%\pagestyle{Standard}
%\newcommand\normalsubformula[1]{\text{\mathversion{normal}$#1$}}
%\title{}
%\author{}
%\date{2014-02-25T11:40:19.646000000}
%\begin{document}
%\clearpage\clearpage\setcounter{page}{1}\pagestyle{MP}
{
\begin{uebsp}
\newcommand\textstyleAbsatzStandardschriftart[1]{#1}
\newcommand\normalsubformula[1]{\text{\mathversion{normal}$#1$}}


%\bigskip

\begin{Exercise}
{\selectlanguage{ngerman}\bfseries
Die Markovkette 
$\normalsubformula{\text{X}}_{\normalsubformula{\text{t}}}\normalsubformula{\text{~}}$
\textstyleAbsatzStandardschriftart{mit Zustandsraum }
$\left(\normalsubformula{\text{0}},\ldots
,\normalsubformula{\text{N}}\right)\normalsubformula{\text{~}}$
\textstyleAbsatzStandardschriftart{hat die
\"Ubergangswahrscheinlichkeiten }\newline
\index{"Ubergangswahrscheinlichkeit@Übergangswahrscheinlichkeit!Beispiel}
\index{R"uckkehrzeit@Rückkehrzeit!Beispiel}

$\normalsubformula{\text{p}}_{\normalsubformula{\text{i}},\normalsubformula{\text{i}}+\normalsubformula{\text{1}}}=\normalsubformula{\text{p}}_{\normalsubformula{\text{i}}},\normalsubformula{\text{~}}\normalsubformula{\text{p}}_{\normalsubformula{\text{i}},\normalsubformula{\text{0}}}=\normalsubformula{\text{1}}-\normalsubformula{\text{p}}_{\normalsubformula{\text{i}}}\normalsubformula{\text{~}}$
\textstyleAbsatzStandardschriftart{mit }
$\normalsubformula{\text{p}}_{\normalsubformula{\text{N}}}=\normalsubformula{\text{0}}\text{.}\normalsubformula{\text{~}}$
\textstyleAbsatzStandardschriftart{Bestimmen Sie die Verteilung der
R\"uckkehrzeit } $\normalsubformula{\text{$\tau
$}}_{\normalsubformula{\text{0}}}\normalsubformula{\text{~}}$
\textstyleAbsatzStandardschriftart{nach 0. Wie muss man die }
$\normalsubformula{\text{p}}_{\normalsubformula{\text{i}}}\normalsubformula{\text{~}}$
\textstyleAbsatzStandardschriftart{w\"ahlen, damit }
$\normalsubformula{\text{t}}_{\normalsubformula{\text{i}}}\normalsubformula{\text{~}}$
\textstyleAbsatzStandardschriftart{auf }
$\left(\normalsubformula{\text{1}},\ldots
,\normalsubformula{\text{N}}+\normalsubformula{\text{1}}\right)\normalsubformula{\text{~}}$
\textstyleAbsatzStandardschriftart{gleichverteilt ist?}}
\end{Exercise}

\begin{Answer}
{\selectlanguage{ngerman}
Skizze}

{\selectlanguage{ngerman}
Die R\"uckkehrwahrscheinlichkeiten sind definiert mit:}

{\centering 
$\normalsubformula{\text{P}}\left(\normalsubformula{\text{$\tau
$}}_{\normalsubformula{\text{0}}}=\normalsubformula{\text{1}}\right)=\left(\normalsubformula{\text{1}}-\normalsubformula{\text{p}}_{\normalsubformula{\text{o}}}\right)$\newline
 $\normalsubformula{\text{P}}\left(\normalsubformula{\text{$\tau
$}}_{\normalsubformula{\text{0}}}=\normalsubformula{\text{2}}\right)=\normalsubformula{\text{p}}_{\normalsubformula{\text{0}}}\left(\normalsubformula{\text{1}}-\normalsubformula{\text{p}}_{\normalsubformula{\text{1}}}\right)$\newline
 $\normalsubformula{\text{P}}\left(\normalsubformula{\text{$\tau
$}}_{\normalsubformula{\text{0}}}=\normalsubformula{\text{3}}\right)=\normalsubformula{\text{p}}_{\normalsubformula{\text{0}}}\normalsubformula{\text{p}}_{\normalsubformula{\text{1}}}(\normalsubformula{\text{1}}-\normalsubformula{\text{p}}_{\normalsubformula{\text{2}}})$\par}

\textstyleAbsatzStandardschriftart{\foreignlanguage{ngerman}{Zur
Erkl\"arung: f\"ur }} $\normalsubformula{\text{$\tau
$}}_{\normalsubformula{\text{0}}}=\normalsubformula{\text{3}}\normalsubformula{\text{~}}$
\textstyleAbsatzStandardschriftart{\foreignlanguage{ngerman}{springe
ich nach drei Schritten zur\"uck zum Anfang. Daf\"ur kann ich nur den
Weg \"uber }}
$\normalsubformula{\text{p}}_{\normalsubformula{\text{0}}},\normalsubformula{\text{~}}{\normalsubformula{\text{~}}\normalsubformula{\text{p}}}_{\normalsubformula{\text{1}}}\normalsubformula{\text{~}}$
\textstyleAbsatzStandardschriftart{\foreignlanguage{ngerman}{und }}
$\left(\normalsubformula{\text{1}}-\normalsubformula{\text{p}}_{\normalsubformula{\text{2}}}\right)\normalsubformula{\text{~}}$
\textstyleAbsatzStandardschriftart{\foreignlanguage{ngerman}{gehen.}}

\begin{equation*}
\normalsubformula{\text{P}}\left(\normalsubformula{\text{$\tau
$}}_{\normalsubformula{\text{0}}}=\normalsubformula{\text{k}}\right)=\normalsubformula{\text{p}}_{\normalsubformula{\text{0}}}\normalsubformula{\text{p}}_{\normalsubformula{\text{1}}}\ldots
\normalsubformula{\text{P}}_{\normalsubformula{\text{k}}-\normalsubformula{\text{2}}}\left(\normalsubformula{\text{1}}-\normalsubformula{\text{p}}_{\normalsubformula{\text{k}}-\normalsubformula{\text{1}}}\right)\normalsubformula{\text{~}}\normalsubformula{\text{~}}\normalsubformula{\text{~}}\normalsubformula{\text{~}}\normalsubformula{\text{~}}\normalsubformula{\text{k}}\in
[\normalsubformula{\text{1}},\normalsubformula{\text{N}}+\normalsubformula{\text{1}}]
\end{equation*}
{\selectlanguage{ngerman}
Damit eine Gleichverteilung vorhanden ist, muss jede R\"uckkehrzeit
gleich wahrscheinlich sein.}

\begin{equation*}
\normalsubformula{\text{P}}\left(\normalsubformula{\text{$\tau
$}}_{\normalsubformula{\text{0}}}=\normalsubformula{\text{1}}\right)=\normalsubformula{\text{P}}\left(\normalsubformula{\text{$\tau
$}}_{\normalsubformula{\text{0}}}=\normalsubformula{\text{2}}\right)=\ldots
=\normalsubformula{\text{P}}\left(\normalsubformula{\text{$\tau
$}}_{\normalsubformula{\text{0}}}=\normalsubformula{\text{k}}\right)=\ldots
=\normalsubformula{\text{P}}(\normalsubformula{\text{$\tau
$}}_{\normalsubformula{\text{0}}}=\normalsubformula{\text{N}}+\normalsubformula{\text{1}})=\frac{\normalsubformula{\text{1}}}{\normalsubformula{\text{N}}+\normalsubformula{\text{1}}}
\end{equation*}
{\selectlanguage{ngerman}
Wir k\"onnen rekursiv beginnen:}

\begin{equation*}
\normalsubformula{\text{P}}(\normalsubformula{\text{$\tau
$}}_{\normalsubformula{\text{0}}}=\normalsubformula{\text{1}})
\end{equation*}
{\centering 
$\left(\normalsubformula{\text{1}}-\normalsubformula{\text{p}}_{\normalsubformula{\text{0}}}\right)=\frac{\normalsubformula{\text{1}}}{\normalsubformula{\text{N}}+\normalsubformula{\text{1}}}$\newline

$\normalsubformula{\text{p}}_{\normalsubformula{\text{0}}}=\normalsubformula{\text{1}}-\frac{\normalsubformula{\text{1}}}{\normalsubformula{\text{N}}+\normalsubformula{\text{1}}}=\frac{\normalsubformula{\text{N}}}{\normalsubformula{\text{N}}+\normalsubformula{\text{1}}}$\par}

\begin{equation*}
\normalsubformula{\text{P}}(\normalsubformula{\text{$\tau
$}}_{\normalsubformula{\text{0}}}=\normalsubformula{\text{2}})
\end{equation*}
{\centering 
$\normalsubformula{\text{p}}_{\normalsubformula{\text{0}}}\left(\normalsubformula{\text{1}}-\normalsubformula{\text{p}}_{\normalsubformula{\text{1}}}\right)=\frac{\normalsubformula{\text{1}}}{\normalsubformula{\text{N}}+\normalsubformula{\text{1}}}$\newline

$\frac{\normalsubformula{\text{N}}}{\normalsubformula{\text{N}}+\normalsubformula{\text{1}}}\left(\normalsubformula{\text{1}}-\normalsubformula{\text{p}}_{\normalsubformula{\text{1}}}\right)=\frac{\normalsubformula{\text{1}}}{\normalsubformula{\text{N}}+\normalsubformula{\text{1}}}$\newline

$\normalsubformula{\text{1}}-\normalsubformula{\text{p}}_{\normalsubformula{\text{1}}}=\frac{\normalsubformula{\text{1}}}{\normalsubformula{\text{N}}}$\newline

$\normalsubformula{\text{p}}_{\normalsubformula{\text{1}}}=\normalsubformula{\text{1}}-\frac{\normalsubformula{\text{1}}}{\normalsubformula{\text{N}}}=\frac{\normalsubformula{\text{N}}-\normalsubformula{\text{1}}}{\normalsubformula{\text{N}}}$\par}

\begin{equation*}
\normalsubformula{\text{P}}(\normalsubformula{\text{$\tau
$}}_{\normalsubformula{\text{0}}}=\normalsubformula{\text{3}})
\end{equation*}
{\centering 
$\normalsubformula{\text{p}}_{\normalsubformula{\text{0}}}\normalsubformula{\text{p}}_{\normalsubformula{\text{1}}}\left(\normalsubformula{\text{1}}-\normalsubformula{\text{p}}_{\normalsubformula{\text{2}}}\right)=\frac{\normalsubformula{\text{1}}}{\normalsubformula{\text{N}}+\normalsubformula{\text{1}}}$\newline

$\frac{\normalsubformula{\text{N}}}{\normalsubformula{\text{N}}+\normalsubformula{\text{1}}}\frac{\normalsubformula{\text{N}}-\normalsubformula{\text{1}}}{\normalsubformula{\text{N}}}\left(\normalsubformula{\text{1}}-\normalsubformula{\text{p}}_{\normalsubformula{\text{2}}}\right)=\frac{\normalsubformula{\text{1}}}{\normalsubformula{\text{N}}+\normalsubformula{\text{1}}}$\newline

$\normalsubformula{\text{1}}-\normalsubformula{\text{p}}_{\normalsubformula{\text{2}}}=\frac{\left(\normalsubformula{\text{N}}+\normalsubformula{\text{1}}\right)\normalsubformula{\text{N}}}{\left(\normalsubformula{\text{N}}+\normalsubformula{\text{1}}\right)\normalsubformula{\text{N}}\left(\normalsubformula{\text{N}}-\normalsubformula{\text{1}}\right)}=\frac{\normalsubformula{\text{1}}}{\normalsubformula{\text{N}}-\normalsubformula{\text{1}}}$\newline

$\normalsubformula{\text{p}}_{\normalsubformula{\text{2}}}=\normalsubformula{\text{1}}-\frac{\normalsubformula{\text{1}}}{\normalsubformula{\text{N}}-\normalsubformula{\text{1}}}=\frac{\normalsubformula{\text{N}}-\normalsubformula{\text{2}}}{\normalsubformula{\text{N}}-\normalsubformula{\text{1}}}$\par}

\textstyleAbsatzStandardschriftart{\foreignlanguage{ngerman}{F\"ur }}
$\normalsubformula{\text{p}}_{\normalsubformula{\text{k}}}\normalsubformula{\text{~}}$
\textstyleAbsatzStandardschriftart{\foreignlanguage{ngerman}{l\"asst
sich definieren:}}

\begin{equation*}
\normalsubformula{\text{p}}_{\normalsubformula{\text{k}}}=\frac{\normalsubformula{\text{N}}-\normalsubformula{\text{k}}}{\normalsubformula{\text{N}}-\normalsubformula{\text{k}}+\normalsubformula{\text{1}}}
\end{equation*}
\end{Answer}
\end{uebsp}
}
