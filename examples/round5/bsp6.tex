% This file was converted to LaTeX by Writer2LaTeX ver. 1.0.2
% see http://writer2latex.sourceforge.net for more info
%\documentclass[a4paper]{article}
%\usepackage[ascii]{inputenc}
%\usepackage[T1]{fontenc}
%\usepackage[english,ngerman]{babel}
%%\usepackage{amsmath}
%\usepackage{amssymb,amsfonts,textcomp}
%\usepackage{color}
%\usepackage{array}
%\usepackage{hhline}
%\usepackage{hyperref}
%\hypersetup{pdftex, colorlinks=true, linkcolor=blue, citecolor=blue, filecolor=blue, urlcolor=blue, pdftitle=, pdfauthor=, pdfsubject=, pdfkeywords=}
% Text styles
%\newcommand\textstyleAbsatzStandardschriftart[1]{#1}
% Page layout (geometry)
%\setlength\voffset{-1in}
%\setlength\hoffset{-1in}
%\setlength\topmargin{2cm}
%\setlength\oddsidemargin{2cm}
%%\setlength\textheight{23.698002cm}
%\setlength\textwidth{17.001cm}
%\setlength\footskip{0.751cm}
%\setlength\headheight{1.251cm}
%\setlength\headsep{0cm}
% Footnote rule
%\setlength{\skip\footins}{0.119cm}
%\renewcommand\footnoterule{\vspace*{-0.018cm}\setlength\leftskip{0pt}\setlength\rightskip{0pt plus 1fil}\noindent\textcolor{black}{\rule{0.25\columnwidth}{0.018cm}}\vspace*{0.101cm}}
% Pages styles
%\makeatletter
%\newcommand\ps@MP{
%  \renewcommand\@oddhead{\"Ubung 5\hfill \hfill 27.11.2013}
%%  \renewcommand\@evenhead{\@oddhead}
%  \renewcommand\@oddfoot{Wahrscheinlichkeitsrechnung\hfill Markus Kessler\hfill Seite %\textstyleAbsatzStandardschriftart{\textbf{\thepage{}}} von %\textstyleAbsatzStandardschriftart{\textbf{?}} \textstyleAbsatzStandardschriftart{und %stochastische Prozesse}}
%%  \renewcommand\@evenfoot{\@oddfoot}
%%  \renewcommand\thepage{\arabic{page}}
%}
%\newcommand\ps@Standard{
%  \renewcommand\@oddhead{}
%  \renewcommand\@evenhead{}
%  \renewcommand\@oddfoot{}
%  \renewcommand\@evenfoot{}
%  \renewcommand\thepage{\arabic{page}}
%}
%\makeatother
%\pagestyle{Standard}
%\newcommand\normalsubformula[1]{\text{\mathversion{normal}$#1$}}
%\title{}
%\author{}
%\date{2014-02-25T11:16:06.549000000}
%\begin{document}
{
\newcommand\textstyleAbsatzStandardschriftart[1]{#1}
\newcommand\normalsubformula[1]{\text{\mathversion{normal}$#1$}}

\begin{uebsp}
\begin{Exercise}

{\selectlanguage{ngerman}\bfseries
In zwei Urnen befinden sich insgesamt N Kugeln. Es wird jeweils eine
Kugel zuf\"allig (gleichverteilt) ausgew\"ahlt und in die andere Urne
gelegt. Die Anzahl 
$\normalsubformula{\text{X}}_{\normalsubformula{\text{t}}}\normalsubformula{\text{~}}$
\textstyleAbsatzStandardschriftart{der Kugeln in Urne 1 nach t
Schritten bildet eine Markovkette. Bestimmen Sie Ihre \"Ubergangsmatrix
und die station\"are Verteilung.}}
\index{"Ubergangsmatrix@Übergangsmatrix!Beispiel}
\index{stationäre Verteilung!Beispiel}

\end{Exercise}
\begin{Answer}

%\bigskip


%\bigskip


%\bigskip


%\bigskip


%\bigskip

{\selectlanguage{ngerman}
F\"ur die \"Ubergangsmatrix werden die Wahrscheinlichkeiten in
Abh\"angigkeit der Gesamtkugeln definiert.}

\begin{equation*}
\normalsubformula{\text{P}}=\left(\begin{matrix}\normalsubformula{\text{0}}&\normalsubformula{\text{1}}&\normalsubformula{\text{0}}&\cdots
&\cdots &\cdots
&\normalsubformula{\text{0}}\\\frac{\normalsubformula{\text{1}}}{\normalsubformula{\text{N}}}&\normalsubformula{\text{0}}&\frac{\normalsubformula{\text{N}}-\normalsubformula{\text{1}}}{\normalsubformula{\text{N}}}&\normalsubformula{\text{0}}&\cdots
&\cdots
&\normalsubformula{\text{0}}\\\normalsubformula{\text{0}}&\frac{\normalsubformula{\text{2}}}{\normalsubformula{\text{N}}}&\normalsubformula{\text{0}}&\frac{\normalsubformula{\text{N}}-\normalsubformula{\text{2}}}{\normalsubformula{\text{N}}}&\normalsubformula{\text{0}}&\cdots
&\normalsubformula{\text{0}}\\\vdots &\ddots &\ddots &\ddots &\ddots
&\ddots &\vdots \\\normalsubformula{\text{0}}&\ddots &\ddots
&\normalsubformula{\text{0}}&\frac{\normalsubformula{\text{N}}-\normalsubformula{\text{1}}}{\normalsubformula{\text{N}}}&\normalsubformula{\text{0}}&\frac{\normalsubformula{\text{1}}}{\normalsubformula{\text{N}}}\\\normalsubformula{\text{0}}&\cdots
&\cdots &\cdots
&\normalsubformula{\text{0}}&\normalsubformula{\text{1}}&\normalsubformula{\text{0}}\end{matrix}\right)
\end{equation*}
\textstyleAbsatzStandardschriftart{\foreignlanguage{ngerman}{Wir bauen
uns nun rekursiv eine Formel f\"ur }} $\normalsubformula{\text{$\pi
$}}_{\normalsubformula{\text{i}}}\normalsubformula{\text{~}}$
\textstyleAbsatzStandardschriftart{\foreignlanguage{ngerman}{auf. Dabei
gibt i die Anzahl der Kugeln in Urne 1 an.}}

Wir versuchen nun im folgenden, $\pi_i$ als Funktion von $\pi_0$ darzustellen:
Beginnen wir bei $\pi_0$: diese kann nur von $\pi_1$ erreicht werden:
\[\pi_0=\frac{1}{N}\pi_1\;\;\Rightarrow\;\;\pi_1=N\cdot \pi_0\]

$\pi_1$ lässt sich nun darstellen als:
\[\pi_1=1\cdot\pi_0+\frac{2}{N}\cdot\pi_2\;\;\Rightarrow\;\;
N\cdot\pi_0-\pi_0=\frac{2}{N}\cdot\pi_2\;\;\Rightarrow\;\;
\pi_2=\frac{N(N-1)}{2}\pi_0\]

$\pi_2$ lässt sich nun darstellen als:
\[\pi_2=\frac{N-1}{N}\pi_1+\frac{3}{N}\pi_3\;\;\Rightarrow\;\;\frac{N(N-1)}{2}\pi_0=\frac{N-1}{\cancel N}\cancel N\cdot \pi_0+\frac{3}{N}\pi_3\;\;\Rightarrow\;\;\]
\[\frac{N(N-1)-2\cdot (N-1)}{2}\cdot \pi_0=\frac{3}{N}\pi_3\;\;\Rightarrow\;\;\pi_3=\frac{(N-2)(N-1)N}{2\cdot 3}\cdot \pi_0\]

\textstyleAbsatzStandardschriftart{\foreignlanguage{ngerman}{F\"ur }}
$\normalsubformula{\text{$\pi
$}}_{\normalsubformula{\text{i}}}\normalsubformula{\text{~}}$
\textstyleAbsatzStandardschriftart{\foreignlanguage{ngerman}{ergibt
das:}}

\begin{equation*}
\normalsubformula{\text{$\pi
$}}_{\normalsubformula{\text{i}}}=\frac{\normalsubformula{\text{N}}\left(\normalsubformula{\text{N}}-\normalsubformula{\text{1}}\right)\left(\normalsubformula{\text{N}}-\normalsubformula{\text{2}}\right)\ldots
\left(\normalsubformula{\text{N}}-\left(\normalsubformula{\text{i}}-\normalsubformula{\text{1}}\right)\right)}{\normalsubformula{\text{i}}!}\normalsubformula{\text{$\pi
$}}_{\normalsubformula{\text{0}}}
\end{equation*}
{\selectlanguage{ngerman}
Die Analogie zum Binomialkoeffizient:}

\begin{equation*}
\left(\begin{matrix}\normalsubformula{\text{n}}\\\normalsubformula{\text{k}}\end{matrix}\right)=\frac{\normalsubformula{\text{n}}!}{\normalsubformula{\text{k}}!\left(\normalsubformula{\text{n}}-\normalsubformula{\text{k}}\right)!}=\frac{\normalsubformula{\text{n}}\left(\normalsubformula{\text{n}}-\normalsubformula{\text{1}}\right)\left(\normalsubformula{\text{n}}-\normalsubformula{\text{2}}\right)\ldots
\left(\normalsubformula{\text{n}}-\left(\normalsubformula{\text{k}}-\normalsubformula{\text{1}}\right)\right)}{\normalsubformula{\text{k}}!}
\end{equation*}
\textstyleAbsatzStandardschriftart{\foreignlanguage{ngerman}{Wir
k\"onnen }} $\normalsubformula{\text{$\pi
$}}_{\normalsubformula{\text{i}}}\normalsubformula{\text{~}}$
\textstyleAbsatzStandardschriftart{\foreignlanguage{ngerman}{umschreiben:}}

\begin{equation*}
\normalsubformula{\text{$\pi
$}}_{\normalsubformula{\text{i}}}=\left(\begin{matrix}\normalsubformula{\text{N}}\\\normalsubformula{\text{i}}\end{matrix}\right)\normalsubformula{\text{$\pi
$}}_{\normalsubformula{\text{0}}}\normalsubformula{\text{~}}\normalsubformula{\text{~}}\normalsubformula{\text{~}}\normalsubformula{\text{~}}\normalsubformula{\text{~}}\normalsubformula{\text{m}}\normalsubformula{\text{i}}\normalsubformula{\text{t}}\normalsubformula{\text{~}}\normalsubformula{\text{i}}\in
[\normalsubformula{\text{0}},\normalsubformula{\text{N}}]
\end{equation*}
Die Summe \"uber alle  $\normalsubformula{\text{$\pi
$}}\normalsubformula{\text{~}}$
\textstyleAbsatzStandardschriftart{beinhaltet alle Wahrscheinlichkeiten
und muss 1 geben.}

\begin{equation*}
\sum
_{\normalsubformula{\text{i}}=\normalsubformula{\text{0}}}^{\normalsubformula{\text{N}}}\normalsubformula{\text{$\pi
$}}_{\normalsubformula{\text{i}}}=\normalsubformula{\text{1}}=\sum
_{\normalsubformula{\text{i}}=\normalsubformula{\text{0}}}^{\normalsubformula{\text{N}}}{\left(\begin{matrix}\normalsubformula{\text{N}}\\\normalsubformula{\text{i}}\end{matrix}\right)\normalsubformula{\text{$\pi
$}}_{\normalsubformula{\text{0}}}}=\normalsubformula{\text{$\pi
$}}_{\normalsubformula{\text{0}}}\sum
_{\normalsubformula{\text{i}}=\normalsubformula{\text{0}}}^{\normalsubformula{\text{N}}}\left(\begin{matrix}\normalsubformula{\text{N}}\\\normalsubformula{\text{i}}\end{matrix}\right)
\end{equation*}
Diese Summe k\"onnen wir durch den Binomischen Lehrsatz(siehe Anhang \ref{sec:binom_lehrsatz}) umschreiben:\index{Binomischer Lehrsatz!Beispiel}

\begin{equation*}
\normalsubformula{\text{$\pi $}}_{\normalsubformula{\text{0}}}\sum
_{\normalsubformula{\text{i}}=\normalsubformula{\text{0}}}^{\normalsubformula{\text{N}}}\left(\begin{matrix}\normalsubformula{\text{N}}\\\normalsubformula{\text{i}}\end{matrix}\right)=\normalsubformula{\text{$\pi
$}}_{\normalsubformula{\text{0}}}\sum
_{\normalsubformula{\text{i}}=\normalsubformula{\text{0}}}^{\normalsubformula{\text{N}}}{\left(\begin{matrix}\normalsubformula{\text{N}}\\\normalsubformula{\text{i}}\end{matrix}\right)\normalsubformula{\text{1}}^{\normalsubformula{\text{N}}-\normalsubformula{\text{1}}}\normalsubformula{\text{1}}^{\normalsubformula{\text{i}}}}=\normalsubformula{\text{$\pi
$}}_{\normalsubformula{\text{0}}}\left(\normalsubformula{\text{1}}+\normalsubformula{\text{1}}\right)^{\normalsubformula{\text{N}}}=\normalsubformula{\text{$\pi
$}}_{\normalsubformula{\text{0}}}\normalsubformula{\text{2}}^{\normalsubformula{\text{N}}}
\end{equation*}
{\selectlanguage{ngerman}
Wie bereits erw\"ahnt muss dies 1 ergeben:}

\begin{equation*}
\normalsubformula{\text{$\pi
$}}_{\normalsubformula{\text{0}}}\normalsubformula{\text{2}}^{\normalsubformula{\text{N}}}=\normalsubformula{\text{1}}\Rightarrow
\normalsubformula{\text{$\pi
$}}_{\normalsubformula{\text{0}}}=\frac{\normalsubformula{\text{1}}}{\normalsubformula{\text{2}}^{\normalsubformula{\text{N}}}}
\end{equation*}
\textstyleAbsatzStandardschriftart{\foreignlanguage{ngerman}{Eingesetzt
in }} $\normalsubformula{\text{$\pi
$}}_{\normalsubformula{\text{i}}}\normalsubformula{\text{~}}$
\textstyleAbsatzStandardschriftart{\foreignlanguage{ngerman}{erhalten
wir die station\"are Verteilung:}}

\begin{equation*}
\normalsubformula{\text{$\pi
$}}_{\normalsubformula{\text{i}}}=\left(\begin{matrix}\normalsubformula{\text{N}}\\\normalsubformula{\text{i}}\end{matrix}\right)\frac{\normalsubformula{\text{1}}}{\normalsubformula{\text{2}}^{\normalsubformula{\text{N}}}}
\end{equation*}
%\end{document}
\end{Answer}
\end{uebsp}
}
