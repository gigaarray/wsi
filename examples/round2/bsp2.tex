\begin{uebsp}
\begin{Exercise}[label=ex:2.2]
Es sei \\ 
\[ F(x) = \left\{
  \begin{array}{l l}
    0 & \quad \text{für $x < 0$}\\
    \frac{{x}^{2}}{4}  & \quad \text{für $0 \le x < 1$} \\
    \frac{x}{2} & \quad \text{für $1 \le x < 2$} \\
    1 & \quad \text{für $x \ge 2$}
  \end{array} \right.\]
\Question
Zeigen Sie, dass $F$ eine Verteilungsfunktion ist.
\Question
$X$ sei nach $F$ verteilt. Bestimmen Sie $\mathbb{P}(X < 1)$, $\mathbb{P}(X \le 1)$, $\mathbb{P}(X = 0$),$\mathbb{P}(X = 1)$, $\mathbb{P}(X = 2)$.
\end{Exercise}
\begin{Answer}
    \index{Verteilungsfunktion!Eigenschaften!Beispiel}
\begin{enumerate}[(a)]
    \item{
        Die Kriterien für eine \reference{Verteilungsfunktion}{Definition}{def:verteilungsfunktion_eigenschaften} lauten:
    	\begin{uebsp_theory}
    	$F:\mathbb R\to\mathbb R$ ist genau dann eine Verteilungsfunktion, wenn
    	\begin{enumerate}
    	\item $0\le F(x)\le 1$ für alle $x$,
    	\item $F$ ist monoton nichtfallend,
    	\item $F$ ist rechtsstetig,
    	\item $\lim_{x\to-\infty}F(x)=0$,
    	\item $\lim_{x\to\infty}F(x)=1$.
    	\end{enumerate}
    	\end{uebsp_theory}
    	
    	Wie wir sehen sind die Punkte alle erfüllgt:
    	
    	\begin{enumerate}
    	\item  \checkmark : Die Werte die $x^{2}$ (bei $0 \le x < 1$) annehmen kann liegen im Bereich [0;1[ selbes gilt für $\frac{x^{2}}{4}$. Auch die Werte für $\frac{x}{2}$ (bei $1 \le x < 2$ ) liegen im Bereich [0,5;1[.
    	\item  \checkmark : Sowohl  $\frac{x^{2}}{4}$ als auch $\frac{x}{2}$ sind monoton steigend.
    	\item \checkmark : erfüllt
    	\item \checkmark : erfüllt
    	\item \checkmark : erfüllt
    	\end{enumerate}
    	
    	
    }
  
\item Diese Wahrscheinlichkeiten lassen sich am besten aus der Verteilungsfunktion berechnen:
    \index{Wahrscheinlichkeiten mit Verteilungsfunktion!Beispiel}
    \begin{uebsp_theory}
        Mit den \reference{Wahrscheinlichkeiten mithilfe der Verteilungsfunktion}{Kapitel}{sec:wahrscheinlichkeiten_verteilungsfunktionen} folgt:
        \[\mathbb P(X\le a)=F_X(a),\]
        \[\mathbb P(X<a)=F_X(a-0),\]
        \[\mathbb P(a< X\le b)=F_X(b)-F_X(a),\]
        \[\mathbb P(a< X< b)=F_X(b-0)-F_X(a),\]
        \[\mathbb P(a\le X\le b)=F_X(b)-F_X(a-0),\]
        \[\mathbb P(a\le X< b)=F_X(b-0)-F_X(a-0).\]
        \[\mathbb P( X= a)=F_X(a)-F_X(a-0).\]
    Dabei ist $F(x-0)=\lim_{h\downarrow 0}F(x-h)$ der linksseitige Grenzwert
    von $F$ in $x$. \\
    \end{uebsp_theory}
    
    Daraus ergibt sich für unser Beispiel: \\
   	\begin{itemize}
   	\item $\mathbb{P}(X < 1)=F_X(1-0)=\frac{1}{4}$
   	\item $\mathbb{P}(X \le 1)=F_X(1)=\frac{1}{2}$
   	\item $\mathbb{P}(X = 0)=F_X(0)-F_X(0-0)=\frac{0^{2}}{4}-0=0$
   	\item $\mathbb{P}(X = 1)=F_X(1)-F_X(1-0)=\frac{1}{2}-\frac{1^{2}}{4}=\frac{1}{4}$
   	\item $\mathbb{P}(X = 2)=F_X(2)-F_X(2-0)=1-\frac{2}{2}=0$
   	\end{itemize}
    
\end{enumerate}
\end{Answer}
\end{uebsp}
