\begin{uebsp}
\begin{Exercise}[label=ex:2.5]
$X$ sei gleichverteilt auf $[0, 1]$. Bestimmen Sie die Verteilung von $-\log X$.
\end{Exercise}
\begin{Answer}
    \index{Gleichverteilung!Beispiel}

Laut Skriptum:\\
Wenn die gemeinsame Verteilung diskret bzw.\ stetig ist, kann man
in dieser Definition die Verteilungsfunktion durch die Wahrscheinlichkeits-
bzw.\ Dichtefunktion ersetzen. 

\begin{uebsp_theory}
    \textbf{Transformationssatz für Dichten}\index{Transformationssatz für Dichten!Beispiel}\\
$X=(X_1,\dots,X_n)$ sei stetig verteilt mit der Dichte $f_X$.
$g:\mathbb R^n\to\mathbb R^n$ sei stetig differenzierbar und 
eindeutig umkehrbar. $Y=g(X)$ (d.h.\ $Y_i=g_i(X_1,\dots, X_n)$)
ist dann ebenfalls stetig verteilt mit der Dichte
\[f_Y(y)=\begin{cases}f_X(g^{-1}(y))|{\partial g^{-1}\over\partial y}(y)|=f_X(g^{-1}(y)){1\over |{\partial g\over\partial x}(g^{-1}(y))|} &\mbox{wenn } y \in g(\mathbb{R}^n), \\
0 & \mbox{sonst.}\end{cases}\]
Dabei ist
\[{\partial g\over\partial x}=\det( ({\partial g_i\over\partial x_j})_{n\times
n})\]
die Funktionaldeterminate.
\end{uebsp_theory}

$Y = g(x) = -log(x)  = -ln(x)\\$
$-Y = ln(x)\\$
$x=e^{-y}=g^{-1}(y)\\$

$F_{Y}(y)=\mathbb{P}(Y \le y)=\mathbb{P}(-ln(x) \le y)=\mathbb{P}(x \ge e^{-y})=e^{-y} \textbf{???}\\\\$

$f_Y(y)=f_X(g^{-1}(y)) \cdot \vert((g^{-1})')\vert=f_X(g^{-1}(y)) \cdot \vert((e^{-y})')\vert=f_X(g^{-1}(y)) \cdot \vert-e^{-y}\vert=\\$
$f_Y(y)=f_X(g^{-1}(y)) \cdot e^{-y}$


\[ f_X(x) = \left\{
  \begin{array}{l l}
    1 & \quad \text{$x \in [0,1]$}\\
    0 & \quad \text{sonst} \\
\end{array} \right.\]

Mit $f_X(x)$ eingesetzt in $f_Y(y)$ folgt: \\

\[ f_Y(y) = \left\{
  \begin{array}{l l}
    e^{-y} & \quad \text{$0 \le g^{-1}(y) \le 1$}\\
    0 & \quad \text{sonst} \\
\end{array} \right.\]

Wobei das aber keine richtige Einschränkung ist, denn $g^{-1}(y)=e^{-y}$ ist sowieso immer kleiner als 1 für $x \in [0,1]$

\end{Answer}
\end{uebsp}
