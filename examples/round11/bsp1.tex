\begin{uebsp}
\begin{Exercise}[label=ex:11.1]
Eine Markovquelle mit drei Zuständen
\begin{align}\begin{pmatrix}
0.7 & 0.2 & 0.1 \\ 
0.1 & 0.7 & 0.2 \\ 
0.1 & 0.3 & 0.6
\end{pmatrix}
\end{align}
Bestimmen Sie die Entropie dieser Quelle.
\end{Exercise}
\begin{Answer}
\index{Entropie!der Quelle!Beispiel}
Zuerst benötigen wir die stationäre Verteilung:\\
$\pi^* = \pi^* P \Rightarrow$
Die Matrix P besitzt zum Eigenwert 1 den Eigenvektor 
$\begin{pmatrix}
1\\
1\\
1
\end{pmatrix}$\\
Zusätzlich gilt:\\
$\pi_1 + \pi_2 + \pi_3 = 1$\\

Ergibt eine stationäre Verteilung von:\\
$\pi^* = \begin{pmatrix}
1/3\\
1/3\\
1/3
\end{pmatrix}$\\

und eine Entropie von:
$H(X) = \sum \pi_i H(P_i)$\\
$\displaystyle = \frac{1}{3} (0.7 log_2(\frac{1}{0.7}) + 0.2 log_2(\frac{1}{0.2}) + 0.1 log_2(\frac{1}{0.1}))$\\


$\displaystyle + \frac{1}{3} (0.1 log_2(\frac{1}{0.1}) + 0.7 log_2(\frac{1}{0.7}) + 0.2 log_2(\frac{1}{0.2}))$\\


$\displaystyle + \frac{1}{3} (0.1 log_2(\frac{1}{0.1}) + 0.3 log_2(\frac{1}{0.3}) + 0.6 log_2(\frac{1}{0.6}))$
\end{Answer}
\end{uebsp}
