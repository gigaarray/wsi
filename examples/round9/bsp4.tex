\begin{uebsp}
\begin{Exercise}[label=ex:9.4]
Gegeben sei folgende Stichprobe:
\[1.5\;2.1\;1.3\;1.7\;2.2\;1.1\;1.9\;0.9\;1.4\;1.6\]
\[1.8\;1.7\;2.3\;1.8\;1.6\;2.0\;1.7\;2.1\;1.8\;1.7\]
einer Normalverteilung. Testen Sie mit dem Chiquadrattest auf Normalverteilung.
\end{Exercise}
\begin{Answer}
\index{Normalverteilung!Beispiel}
\index{Chi-Quadrat-Test!Beispiel}
\begin{uebsp_theory}
Wenn nicht alle Parameter aus der Verteilung bekannt sind, müssen diese mit der ML-Methode geschätzt werden. \reference{}{Satz}{satz:verteilung_nicht_vollstaendig}
\end{uebsp_theory}
\begin{enumerate}[i)]
\item \textbf{Schätzen von $\mu$}:\\
Für $\hat\mu_n$ gilt: \[\hat\mu_n=\overline x_n=\frac{1}{n}\sum_{i=1}^nx_i=\frac{1}{20}\sum_{i=1}^{20}x_i=\frac{34.2}{20}=1.71\]
\item \textbf{Schätzen von $\sigma^2$}:
\begin{uebsp_theory}
\reference{Stichprobenvarianz für ML-Schätzer}{Definition}{def:stichprobenvarianz}
\[s_n^2= \frac{1}{n} \sum_{i=1}^n\left(x_i-\overline x\right)^2\]
\end{uebsp_theory}
\[s_n^2= \frac{1}{20} \sum_{i=1}^{20}(x_i-1.71)^2=0.12\]
\item \textbf{Klasseneinteilung}:\\
Wir teilen die Stichproben auf $k=5$ Klassen auf:
\[k_1=[0-20\%],\;k_2=[20-40\%],\;k_3=[40-60\%],\;k_4=[60-80\%],\;k_5=[80-100\%]\]
ausgehend von Standardnormalverteilung erhalten wir für somit für $z_{p_i}$: (die Werte aus \reference{Tabelle}{Anhang}{sec:quantile_chi_verteilung} lesen)
\[z_{20\%}=-0.842,z_{40\%}=-0.253,z_{60\%}=0.253,z_{80\%}=0.842\].

Umgerechnet auf die Normalverteilung mit den vorher geschätzen Parametern $\hat\mu$ und $s_n^2$ ergeben sich somit folgende Werte \reference{}{Definition}{def:transformation_normalverteilung}:
\[\tau_{20}=\mu+\sigma\cdot z_{20}=1.71+0.346\cdot(-0.842)=1.42\]
\[\tau_{40}=\mu+\sigma\cdot z_{40}=1.71+0.346\cdot(-0.253)=1.62\]
\[\tau_{60}=\mu+\sigma\cdot z_{60}=1.71+0.346\cdot(0.253)=1.80\]
\[\tau_{80}=\mu+\sigma\cdot z_{80}=1.71+0.346\cdot(0.842)=2.00\]
Nun können wir die einzelnen Stichprobenwerte in die Klassen einsortieren:
\begin{center}
\begin{tabular}{|c|c|c|c|c|c|}
\hline
Klasse & 1 & 2 & 3 & 4 & 5\\
\hline
Untere Schranke & $-\infty$ &  1.42 & 1.62 & 1.8 & 2\\
\hline
Obere Schranke & 1.42 & 1.62 & 1.8 & 2 & $\infty$\\
\hline
Anzahl & 5 & 2 & 4 & 3 & 6\\
\hline
\end{tabular}
\end{center}

Wir erwarten aber Elemente in Klasse laut Verteilungsfunktion\\
$n_i=p_i\cdot n=\frac{1}{5}\cdot 20=4$
Stichproben.

\item \textbf{Bestimmen der Statistik $T$}
\begin{uebsp_theory}
Die Statistik $T$(Gewichtete Quadratsumme) ist definiert, als
\[T=\sum_{i=1}^k\frac{(Y_i-np_i)^2}{np_i}\]
\end{uebsp_theory}
Somit erhalten wir für die Statistik $T:$\\
\[T=\sum_{i=1}^5\frac{(n_i-4)^2}{4}=\frac{(5-4)^2}{4}+\frac{(2-4)^2}{4}+\frac{(4-4)^2}{4}+\frac{(3-4)^2}{4}+\frac{(6-4)^2}{4}\]
\[T=\frac{5}{2}\]

\item \textbf{$\chi^2_{n-1-d,1-\alpha}$ aus der Tabelle bestimmen:}
\begin{uebsp_theory}
Da wir 2 Parameter($d=2$) geschätzt haben, muss die Anzahl der Freiheitsgrade korrigiert werden: $k-1-d=5-1-2=2$.
Somit müssen wir statt $k-1$ -Freiheitsgraden $k-1-d$ -Freiheitsgrade verwenden. Und in der \reference{Tabelle}{Anhang}{sec:quantile_chi_verteilung} nach $\chi^2_{n-1-d,1-\alpha}$ suchen.
\end{uebsp_theory}
\[\chi^2_{2,0.95}=5.99\]

\item \textbf{Ungleichung lösen}:\\
Wir müssen $H_0$ verwerfen, wenn $T>\chi^2$:
\[2.5=T>\chi^2=5.99\]
$\Rightarrow H_0$ annhemen, $\Rightarrow$ Stichprobe ist normalverteilt.
\end{enumerate}
\end{Answer}
\end{uebsp}
