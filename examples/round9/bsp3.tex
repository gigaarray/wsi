\begin{uebsp}
\begin{Exercise}[label=ex:9.3]
Gegeben sei folgende Stichprobe:
\[1.5\;2.1\;1.3\;1.7\;2.2\;1.1\;1.9\;0.9\;1.4\;1.6\]
\[1.8\;1.7\;2.3\;1.8\;1.6\;2.0\;1.7\;2.1\;1.8\;1.7\]
einer Normalverteilung. Testen Sie $H_0:\sigma_0^2\leq 0.1$ gegen $H_1:\sigma^2>0.1$.
\end{Exercise}
\begin{Answer}
\index{Spezielle Tests!Beispiel}
\index{Tests!für $\sigma$ von $\mathcal N$!Beispiel}
\index{Korrigierte Stichprobenvarianz!Beispiel}
\begin{uebsp_theory}
\reference{Test für die Varianz $\sigma^2$ (für Normalverteilung)}{Kapitel}{sec:spezielle_tests}
\[T=\frac{S_n^2(n-1)}{\sigma_0^2}\]
Für $H_0:\sigma^2 \leq\sigma_0^2$ gegen $H_1:\sigma^2>\sigma_0^2$:
    verwerfen $H_0$, wenn $T>\chi^2_{n-1;1-\alpha}$

Wobei $S_n^2$ definiert ist, als die \reference{korrigierte Stichprobenvarianz}{Definition}{def:korrigierte_stichprobenvarianz}:
\[s_n^2=\frac{1}{n-1}\sum_{i=1}^n(x_i-\overline x)^2\]
\end{uebsp_theory}
\begin{enumerate}[i)]
\item \textbf{Berechnung des Mittelwertes:}
\[\overline x_n=\frac{1}{n}\sum_{i=1}^nx_i=\frac{1}{20}\sum_{i=1}^{20}x_i=\frac{34.2}{20}=1.71\]
\item \textbf{Berechnung der korrigierten Stichprobenvarianz:}
\begin{eqnarray*}
s_n^2&=&\frac{1}{19}\sum_{i=1}^{20}(x_i-1.71)^2=\frac{1}{19}((1.5-1.71)^2+(2.1-1.71)^2+...+(1.7-1.71)^2)\\
s_n^2&=&\frac{2.398}{19}\approx0.1262
\end{eqnarray*}
\item \textbf{Berechnung von $T:$}
\[T=\frac{2.398\cdot(20-1)}{19\cdot \sigma_0^2}=\frac{2.398}{0.1}=23.98>\chi^2_{n-1;1-\alpha}\]
\item \textbf{$t_{n-1,1-\alpha/2}$ aus der Tabelle bestimmen:}
Für \reference{Signifikanzniveau}{Definition}{def:signifikanzniveau} $\alpha=0.5$:
\[\chi^2_{20-1,1-5}=\chi^2_{19,0.95}=30.144\;\leftarrow\framebox{aus \reference{Tabelle}{Anhang}{sec:quantile_chi_verteilung}}\]
\item \textbf{Ungleichung lösen}
\[23.98=T>\chi^2_{19,0.95}=30.144\;\Rightarrow\;\framebox{können $H_0$ nicht verwerfen.}\]
\end{enumerate}
\end{Answer}
\end{uebsp}
