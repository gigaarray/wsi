\begin{uebsp}
\begin{Exercise}[label=ex:10.2]
\index{Mittlere Unbestimmtheit!Beispiel}
\index{Entropie!Beispiel}
Vergleichen Sie im vorigen Beispiel \ref{ex:10.1} die mittlere Unbestimmtheit und die
Entropie.
\end{Exercise}
\begin{Answer}
\textbf{Mittlere Unbestimmtheit:}\\
$\displaystyle H^*(P) = \sum_{i=1}^{4}p_i l_i$\\
$\displaystyle H^*(P) =  0.4 \cdot 1 + 0.25 \cdot 2 + 0.2 \cdot 3 + 0.15 \cdot 4 = 2.1$\\
\textbf{Entropie}
$\displaystyle H(P) = \sum_{i=1}^{4} p_i log_2(\frac{1}{p_i})$\\
$\displaystyle H(P) = 0.1 \cdot log_2(\frac{1}{0.1}) + 0.4 \cdot log_2(\frac{1}{0.4}) + 0.05 \cdot log_2(\frac{1}{0.05}) 
+ 0.2 \cdot log_2(\frac{1}{0.2}) + 0.15 \cdot log_2(\frac{1}{0.15})$\\
$H(P) = 1.9$
\end{Answer}
\end{uebsp}
