\begin{uebsp}
\begin{Exercise}[label=ex:10.4]
Bestimme für die Verteilung: 
\begin{align}
P = (0.1, 0.4, 0.05, 0.2, 0.25)
\end{align}
den Fanu-Code
\end{Exercise}
\begin{Answer}
\index{Fanu-Code!Beispiel}
Wir brauchen die Wahrscheinlichkeiten nicht sortieren, sondern können gleich beginnen die Codewortlängen, die Verteilung, sowie \[\frac{(f_{i-1}+f_i)}{2}\] auszurechnen.\\
\[l_i = \lceil log_2(\frac{1}{p_i}) \rceil + 1\]
\[f_i = \sum_{j=1}^{i} p_j\]
\begin{center}
\begin{tabular}{|c|c|c|c|c|}
\hline Wahrscheinlichkeit $p_i$ & Codewortlänge $l_i$ & $f_i$ & $f_{i-1}$ & $\frac{(f_{i-1}+f_i)}{2}$ \\ 
\hline 0.1 & 5 & 0.1 & 0 & 0.05 \\
\hline 0.4 & 3 & 0.5 & 0.1 &  0.3  \\ 
\hline 0.05 & 6 & 0.55 & 0.5 & 0.525 \\ 
\hline 0.2 & 4 & 0.75 & 0.55 & 0.65 \\ 
\hline 0.25 & 3 & 1 & 0.75 & 0.875 \\ 
\hline 
\end{tabular}
\end{center}
Die Codewörter $c_i$ erhält man durch umwandeln von
\[\displaystyle\frac{(f_{i-1}+f_i)}{2}\] in Binärdarstellung.\\
In unserem Fall ergibt das:\\
\begin{center}
\begin{tabular}{|c|c|c|c|c|c|}
\hline Wahrscheinlichkeit $p_i$ & Codewortlänge $l_i$ & $f_i$ & $f_{i-1}$ & $\displaystyle \frac{(f_{i-1}+f_i)}{2}$ & $c_i$ \\ 
\hline 0.1 & 5 & 0.1 & 0 & 0.05 & 00001\\ 
\hline 0.4 & 3 & 0.5 & 0.1 & 0.3 & 010 \\ 
\hline 0.05 & 6 & 0.55 & 0.5 & 0.525 & 100001\\ 
\hline 0.2 & 4 & 0.75 & 0.55 & 0.65 & 1010\\ 
\hline 0.25 & 3 & 1 & 0.75 & 0.875 & 111\\ 
\hline 
\end{tabular}
\end{center}
\index{Mittlere Unbestimmtheit!Beispiel}
Als mittlere Codewortlänge ergibt sich \[H^*(P)=\mathbb{E}(l) = \sum_{i=1}^{5}p_i\cdot l_i = 0.1\cdot 5+0.4\cdot 3+0.05\cdot 3+0.05\cdot 6+0.2\cdot 4+0.25\cdot 3=3.55\]
\end{Answer}
\end{uebsp}
