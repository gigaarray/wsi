\begin{uebsp}
\begin{Exercise}[label=ex:10.5]
Bestimmen Sie für $m = 1, 2, 3, 4$ geschlossene Ausdrücke für die mittlere Unbestimmtheit $H^*(P)$.
\end{Exercise}
\begin{Answer}
\index{Mittlere Unbestimmtheit!Beispiel}
\textbf{Für m = 1:}\\
Für einen Binärbaum mit nur einem Blatt ergibt
\[H^*(P) = l_1 p_1\]
mit $l_1 = 0$ und $p_1 = 1$ 
eine mittlere Unbestimmtheit von:
\[H^*(P) = 0\]


\textbf{Für m = 2:}\\
Ein Binärbaum mit m = 2 hat die Form:
\begin{center}
\begin{tikzpicture}[every tree node/.style={draw,circle},
   level distance=1.3cm,sibling distance=.8cm, anchor=west,
    minimum size=8mm,
   edge from parent path={(\tikzparentnode) -- (\tikzchildnode)}]
	\Tree [.\node (A) {} ; 
		[.\node (B) {1};]
		[.\node (C) {2};]
		]
\end{tikzpicture}
\end{center}
Daraus ergibt sich eine mittlere Unbestimmtheit von:
\[H^*(P) = l_1 p_1 + l_2 p_2\]
Da $l_1 = l_2 = 1$ ist und die Summe der Wahrscheinlichkeiten 1 sein muss, folgt:
\[H^*(P) = p_1 + p_2 = 1\]


\textbf{Für m = 3:}\\
o.B.d.A sei: 
$p_1 \le p_2 \le p_3$, \\
daraus bildet sich ein Binärbaum:
\begin{center}
\begin{tikzpicture}[every tree node/.style={draw,circle},
   level distance=1.3cm,sibling distance=.8cm, anchor=west,
    minimum size=8mm,
   edge from parent path={(\tikzparentnode) -- (\tikzchildnode)}]
	\Tree [.\node (A) {} ; 
		[.\node (B) {1};]
		[.\node (C) {};
			[.\node (D) {2};]
			[.\node (E) {3};]
		]
	]
\end{tikzpicture}
\end{center}
Für die mittlere Unbestimmtheit heißt, dass:
\[H^*(P) = l_1 p_1 + l_2 p_2 + l_3 p_3 = p_1 + 2(p_2 + p_3)\]
Da $p_1 + p_2 + p_3 = 1$ ist, folgt daraus:
\[H^*(P) = p_1 + 2(1 - p_1) = 2 - p1\]
bzw.
\[H^*(P) = 1 - (p_2 + p_3) + 2(p_2 + p_3) = 1 + p_2 + p_3\]


\textbf{Für m = 4:}\\
o.B.d.A sei: 
$p_1 \le p_2 \le p_3 \le p_4$, \\
dabei sind 2 verschiedene Binärbäume möglich:
\begin{center}
\begin{tikzpicture}[every tree node/.style={draw,circle},
   level distance=1.3cm,sibling distance=.8cm, anchor=west,
    minimum size=8mm,
   edge from parent path={(\tikzparentnode) -- (\tikzchildnode)}]
	\Tree [.\node (A) {} ; 
		[.\node (B) {};
				[.\node (C) {1};]
				[.\node (D) {2};]
		]
		[.\node (E) {};
			[.\node (F) {3};]
			[.\node (G) {4};]
		]
	]
\end{tikzpicture}
\end{center}
mit einer mittleren Unbestimmtheit von:
\[H_1^*(P) = 2(p_1 + p_2 + p_3 + p_4) = 2\]
Bzw.
\begin{center}
\begin{tikzpicture}[every tree node/.style={draw,circle},
   level distance=1.3cm,sibling distance=.8cm, anchor=west,
    minimum size=8mm,
   edge from parent path={(\tikzparentnode) -- (\tikzchildnode)}]
	\Tree [.\node (A) {} ; 
		[.\node (B) {1};	]
		[.\node (C) {};
			[.\node (D) {2};]
			[.\node (E) {};
				[.\node (F) {3};]
				[.\node (G) {4};]]
		]
	]
\end{tikzpicture}
\end{center}
mit einer mittleren Unbestimmtheit von:
\[H_2^*(P) = p_1 + 2 p_2 + 3 (p_3 + p_4) = p_1 + 2 p_2 + 3(1- (p_1 + p_2))\]
\[= p_1 + 2 p_2 + 3 - 3 p_1 - 3 p_2 = 3 - 2 p_1 - p_2\]

Da je nach Verteilung der Binärbaum mit der kleineren mittleren Unbestimmtheit verwendet wird, ergibt das insgesamt:
\[H^*(P) = \min(2, 3 - 2 p_1 - p_2)\]
\end{Answer}
\end{uebsp}
