\begin{uebsp}
\begin{Exercise}[label=ex:10.3]
Bestimme für die Verteilung: 
\begin{align}
P = (0.1, 0.4, 0.05, 0.2, 0.25)
\end{align}
den Shannon-Code
\end{Exercise}
\begin{Answer}
\index{Shannon-Code!Beispiel}
Zuerst ordnen wir die Verteilung absteigend:\\
\[P = (0.4, 0.25, 0.2, 0.1, 0.05)\]
Als nächstes bestimmen wir die Codewortlängen 
\[l_i = \lceil log_2(\frac{1}{p_i}) \rceil\]
sowie die Verteilung 
\[f_i = \sum_{j=1}^{i} p_j\]
\begin{center}
\begin{tabular}{|c|c|c|}
\hline Wahrscheinlichkeit $p_i$ & Codewortlänge $l_i$ & Verteilung $f_{i-1}$ \\ 
\hline 0.4 & 2 & 0 \\ 
\hline 0.25 & 2 & 0.4 \\ 
\hline 0.2 & 3 & 0.65 \\ 
\hline 0.1 & 4 & 0.85 \\ 
\hline 0.05 & 5 & 0.9 \\ 
\hline 
\end{tabular}
\end{center}
Die Codewörter $c_i$ erhält man durch umwandeln von $f_{i-1}$ in Binärdarstellung.\\
In unserem Fall ergibt das:
\begin{center}
\begin{tabular}{|c|c|c|c|}
\hline Wahrscheinlichkeit $p_i$ & Codewortlänge $l_i$ & Verteilung $f_{i-1}$ & Codewort $c_i$\\ 
\hline 0.4 & 2 & 0 & 00\\ 
\hline 0.25 & 2 & 0.4 & 01\\ 
\hline 0.2 & 3 & 0.65 & 101\\ 
\hline 0.1 & 4 & 0.85 & 1101\\ 
\hline 0.05 & 5 & 0.9 & 11101\\ 
\hline 
\end{tabular}
\end{center}
\index{Mittlere Unbestimmtheit!Beispiel}
Als mittlere Codewortlänge ergibt sich 
\[H^*(P)=\mathbb{E}(l) = \sum_{i=1}^{5}p_i\cdot l_i = 0.4\cdot 2+0.25\cdot 2+0.2\cdot 3+0.1\cdot 4+0.05\cdot 5 = 2.55\]
\end{Answer}
\end{uebsp}
