\begin{uebsp}
\begin{Exercise}[label=ex:3.5]
Bei einem Spiel kann auf die Ausgänge $1, ..., m$ gesetzt werden, die mit Wahrscheinlichkeiten $p_1 , ..., pm$ gezogen werden. Wenn Ausgang $i$ gezogen wird, werden die Einsätze auf $i$ $m$-fach zurückgezahlt, die anderen verfallen. Ein Spieler spielt nach folgender Strategie: er verteilt sein Kapital $K$ im Verhältnis $q_1: ...: q_m$ (mit $\sum_i q_i=1$) auf die möglichen Ausgänge und verwendet den Gewinn aus einer Runde als Einsatz in der nächsten.
\Question
Zeigen Sie, dass das Kapital nach $n$ (unabhängigen) Runden 
\[K_n=K_0X_1...X_n\]
ist, mit $\mathbb{P}(X_i=mq_j)=p_j$.
\Question
Bestimmen Sie
\[\lim_{n\to\infty}\frac{1}{n}log(K_n)\]
\Question
Wie sind $q_1,...,q_m$ zu wählen, damit dieser Grenzwert maximal wird?
\end{Exercise}
\begin{Answer}
%TODO!!!!!!
\end{Answer}
\end{uebsp}
