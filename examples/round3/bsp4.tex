\begin{uebsp}
\begin{Exercise}[label=ex:3.4]
Bestimmen Sie Erwartungswert und Varianz der geometrischen Verteilung.
\end{Exercise}
\begin{Answer}
    \begin{enumerate}[i)]
        \item Erwartungswert:
            \begin{uebsp_theory}
                Der \reference{Erwartungswert für diskrete Zufallsvariablen}{Kapitel}{sec:erwartungswert} ist mit \index{Erwartungswert!Beispiel}
                    \[\mathbb E(X)=\sum_x xp_X(x)\]
                definiert
            \end{uebsp_theory}
            \begin{uebsp_theory}
                Die \reference{Geometrische Verteilung}{Kapitel}{sec:geometrische_verteilung} $\mathcal G(p)$: \index{Geometrische Verteilung!Beispiel}
                    \[f(x) = \begin{cases} 
                                p(1-p)^x &\mbox{wenn } 0\leq x \\
                                0 & \mbox{sonst}
                \end{cases}\]
            \end{uebsp_theory}
            \begin{uebsp_theory}
                Die geometrische Reihe:
                \[\sum_{i=0}^\infty i\cdot q^i=\frac{q}{(1-q)^2}\;\;\;\;\forall |q|<1\]
                %TODO: copy this definition to scriptum
                %TODO: don't forget to index as example
            \end{uebsp_theory}
            In Form $\sum x\cdot \alpha^x$ (die der Geometrischen Reihe) bringen:\\
            \begin{eqnarray*}
                \mathbb{E}(X) &=& \sum_{x=1}^{\infty}x\cdot p(1-p)^{x-1}=p\cdot \sum_{x=1}^{\infty}x\cdot (1-p)^{x-1}=\frac{p}{1-p}\sum_{x=1}^{\infty}x\cdot (1-p)^x \\
                 &=& \frac{p}{1-p}\sum_{x=1}^{\infty}x\cdot (1-p)^x = \frac{p}{1-p}\cdot\frac{1-p}{(1-(1-p))^2}=\frac{p}{\cancel{1-p}}\cdot\frac{\cancel{1-p}}{(1-1+p)^2}=\\
                \mathbb{E}(X) &=&\frac{p}{p^2}=\frac{1}{p} = \frac{1}{p}
            \end{eqnarray*}
        \item Varianz:
            \begin{uebsp_theory}
                Die \reference{Varianz}{Kapitel}{sec:varianz} ist mit \index{Varianz!Beispiel}
                \[\mathbb V(X)=\mathbb E((X-\mathbb E(x))^2)=\mathbb E(X^2)-(\mathbb E(x))^2)\]
                definiert.
                %TODO: 2.te herleitung mittels verschiebungssatz ins scriptum evtl. aufnehmen.
            \end{uebsp_theory}
            \begin{eqnarray*}
                \mathbb V(X)&=&\mathbb E(X^2)-(\mathbb E(x))^2)
            \end{eqnarray*}

            mit $\mathbb{E}(X)^2=\frac{1}{p^2}$ ist der eine Teil bereits bekannt. Der 2. Teil $\mathbb{E}(X^2)$ ist gesucht:

            \begin{uebsp_theory}
                Sonderform der geometrischen Reihe:
                \[\sum_{i=0}^\infty i^2\cdot q^i=\frac{q(1+q)}{(1-q)^3}\;\;\;\;\forall |q|<1\]
                %TODO: copy this definition to scriptum
                %TODO: don't forget to index as example
            \end{uebsp_theory}

            Wir versuchen, $\mathbb{E}(X^2)$ in die Form $\sum x^2\cdot \alpha^x$ (Sonderform der Geometrischen Reihe) zu bringen.

            \begin{eqnarray*}
                \mathbb{E}(X^2) &=& \sum_{x=1}^{\infty}x^2\cdot p(1-p)^{x-1}=p\sum_{x=1}^{\infty}x^2\cdot (1-p)^{x-1}=\frac{p}{1-p}\sum_{x=1}^{\infty}x^2\cdot (1-p)^{x} \\
                \mathbb{E}(X^2) &=& \frac{p}{1-p}\sum_{x=1}^{\infty}x^2\cdot (1-p)^{x} = \frac{p}{\cancel{1-p}}\frac{\cancel{(1-p)}(1+1-p)}{(1-(1-p))^3}=\frac{p\,(2-p)}{p^3}=\frac{2-p}{p^2}
            \end{eqnarray*}

            \begin{eqnarray*}
                \mathbb V(X)&=&\mathbb E(X^2)-(\mathbb E(x))^2) = \frac{2-p}{p^2}-\frac{1}{p^2}=\frac{1-p}{p^2}
            \end{eqnarray*}
    \end{enumerate}
\end{Answer}
\end{uebsp}
