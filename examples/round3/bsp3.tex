\begin{uebsp}
\begin{Exercise}[label=ex:3.3]
Bestimmen Sie Erwartungswert und Varianz der Betaverteilung.
\end{Exercise}
\begin{Answer}
    \begin{enumerate}[i)]
        \item Erwartungswert:
            \begin{uebsp_theory}
                Der \reference{Erwartungswert für stetige Zufallsvariablen}{Kapitel}{sec:erwartungswert} ist mit \index{Erwartungswert!Beispiel}
                    \[\mathbb E(X)=\int_{-\infty}^\infty xf_X(x)dx\]
                definiert
            \end{uebsp_theory}

            \begin{uebsp_theory}
                Die \reference{Betaverteilung 1. Art}{stetige Verteilungen}{sec:betaverteilung_first} $\beta(\alpha, \lambda)$: \index{Betaverteilung!Beispiel}
                    \[f(x) = \begin{cases} 
                                \frac{(1-x)^{\beta-1} x^{\alpha-1}}{\beta(\alpha, \beta)} &\mbox{wenn } 0\leq x \leq 1 \\
                                0 & \mbox{sonst}
                                \end{cases}\]
            \end{uebsp_theory}
            \begin{eqnarray*}
                \mathbb{E}(x) &=& \int_0^1 x\cdot f_X(x)dx\;\;\fbox{denn Betaverteilung nur im Bereich $0\leq x\leq1$ def. } \\
                 &=& \int_0^1 x\cdot \frac{(1-x)^{\beta-1} x^{\alpha-1}}{\beta(\alpha, \beta)}dx\\
                 &=& \frac{1}{\beta(\alpha, \beta)} \int_0^1 (1-x)^{\beta-1} \cdot x^{\alpha}dx\;\;\fbox{ substituiere $\Delta=\alpha+1\Rightarrow\alpha=\Delta-1$}\\
                \mathbb{E}(x) &=& \frac{1}{\beta(\alpha, \beta)} \int_0^1 (1-x)^{\beta-1} \cdot x^{\Delta-1}dx
            \end{eqnarray*}

            \begin{uebsp_theory}
                Die Betafunktion ist definiert als:
                %TODO: copy this definition to scriptum
                %TODO: don't forget to index as example
                \[\beta(x,y)=\int_0^1t^{x-1}(1-t)^{y-1}dt=\frac{\Gamma(x)\cdot\Gamma(y)}{\Gamma(x+y)}\]
            \end{uebsp_theory}

            \begin{eqnarray*}
                \mathbb{E}(x) &=& \frac{1}{\beta(\alpha, \beta)} \int_0^1 (1-x)^{\beta-1} \cdot x^{\Delta-1}dx\\
                 &=& \frac{1}{\beta(\alpha, \beta)} \beta(\Delta, \beta) = \frac{\beta(\Delta, \beta)}{\beta(\alpha, \beta)}\;\;\fbox{rücksubstituieren: $\Delta=\alpha+1$}\\
                 &=& \frac{\beta(\alpha+1, \beta)}{\beta(\alpha, \beta)}\;\;\fbox{ mit der Betafunktion eingesetzt:}\\
                \mathbb{E}(x) &=& \frac{\Gamma(\alpha+1)\cdot\cancel{\Gamma(\beta)}}{\Gamma(\alpha+1+\beta)} \cdot \frac{\Gamma(\alpha+\beta)}{\Gamma(\alpha)\cdot\cancel{\Gamma(\beta)}}\;\;\fbox{ mit der Gammafktn. eingesetzt:}
            \end{eqnarray*}

            \begin{uebsp_theory}
                Die Gammafunktion ist definiert als:
                %TODO: copy this definition to scriptum
                %TODO: don't forget to index as example
                \[\Gamma(x)=\int_0^\infty x^{t-1}e^{-x}dx=(z-1)\Gamma(z-1)\]
            \end{uebsp_theory}
            \begin{eqnarray*}
                \mathbb{E}(x) &=& \frac{\Gamma(\alpha+1)}{\Gamma(\alpha+1+\beta)} \cdot \frac{\Gamma(\alpha+\beta)}{\Gamma(\alpha)} = \frac{\cancel{\Gamma(\alpha)}\cdot\alpha}{\Gamma(\alpha+1+\beta)} \cdot \frac{\Gamma(\alpha+\beta)}{\cancel{\Gamma(\alpha)}}\\
                 &=& \frac{\alpha\cdot \Gamma(\alpha+\beta)}{\Gamma(\alpha+1+\beta)} =  \frac{\alpha\cdot \cancel{\Gamma(\alpha+\beta)}}{\cancel{\Gamma(\alpha+\beta)}\cdot(\alpha+\beta)} = \frac{\alpha}{\alpha+\beta}\\
                \mathbb{E}(x) &=& \frac{\alpha}{\alpha+\beta}
            \end{eqnarray*}

        \item Varianz:
            \begin{uebsp_theory}
            Die \reference{Varianz}{Kapitel}{sec:varianz} ist mit \index{Varianz!Beispiel}
                \[\mathbb{V}(x)=\int_{-\infty}^\infty (x-\mathbb{E}(x))^2\cdot f_X(x)dx\]
                %TODO: Diese Varianzformel (kommt aus wiki) ebenfalls ins skriptum übernehmen.
            definiert.
            \end{uebsp_theory}

            \begin{eqnarray*}
                \mathbb{V}(x) &=& \int_{0}^1 (x-\mathbb{E}(x))^2\cdot f_X(x)\cdot dx = \int_{0}^1 \left(x-\frac{\alpha}{\alpha+\beta}\right)^2\cdot \frac{(1-x)^{\beta-1} x^{\alpha-1}}{\beta(\alpha, \beta)}\cdot dx \\
                 &=& \frac{1}{\beta(\alpha, \beta)}\cdot \int_{0}^1 \left(x^2-\frac{2\,\alpha\,x}{\alpha+\beta} + \frac{\alpha^2}{(\alpha+\beta)^2}\right)\cdot (1-x)^{\beta-1} x^{\alpha-1}\cdot dx \\
                 &=& \frac{1}{\beta(\alpha, \beta)}\cdot \int_{0}^1\left((1-x)^{\beta-1} x^{\alpha+1} - \frac{2\,\alpha\,(1-x)^{\beta-1}\,x^{\alpha}}{\alpha+\beta} + \frac{\alpha^2\,(1-x)^{\beta-1}\,x^{\alpha-1}}{(\alpha+\beta)^2} \right)\cdot dx \\
                 &=& \frac{1}{\beta(\alpha, \beta)}\cdot \left(\int_{0}^1(1-x)^{\beta-1} x^{\alpha+1}\cdot dx - \int_{0}^1\frac{2\,\alpha\,(1-x)^{\beta-1}\,x^{\alpha}}{\alpha+\beta}\cdot dx + \int_{0}^1\frac{\alpha^2\,(1-x)^{\beta-1}\,x^{\alpha-1}}{(\alpha+\beta)^2}\cdot dx \right)\\
                \mathbb{V}(x) &=& \frac{1}{\beta(\alpha, \beta)}\cdot \left(NR1 - NR2 + NR3 \right)
            \end{eqnarray*}

            \begin{multicols}{2}
                \begin{eqnarray*}
                    NR1 &=& \int_{0}^1(1-x)^{\beta-1} x^{\alpha+1}\cdot dx\\
                     &=& \int_{0}^1(1-x)^{\beta-1} x^{\Delta-1}\cdot dx\\
                     &=& \frac{\Gamma(\Delta)\Gamma(\beta)}{\Gamma(\Delta+\beta)} = \frac{\Gamma(\alpha+2)\Gamma(\beta)}{\Gamma(\alpha+2+\beta)}\\
                     &=& \frac{(\alpha+1)\Gamma(\alpha+1)\Gamma(\beta)}{\Gamma(\alpha+1+\beta)(\alpha+1+\beta)}\\
                    NR1 &=& \frac{\alpha(\alpha+1)\Gamma(\alpha)\Gamma(\beta)}{\Gamma(\alpha+\beta)(\alpha+1+\beta)(\alpha+\beta)}\\\\\\
                    NR3 &=& \int_{0}^1\frac{\alpha^2\,(1-x)^{\beta-1}\,x^{\alpha-1}}{(\alpha+\beta)^2}\cdot dx\\
                     &=& \frac{\alpha^2}{(\alpha+\beta)^2}\cdot \int_{0}^1 (1-x)^{\beta-1}\,x^{\alpha-1}\cdot dx\\
                    NR3 &=& \frac{\alpha^2}{(\alpha+\beta)^2} \frac{\Gamma(\alpha)\Gamma(\beta)}{\Gamma(\alpha+\beta)}\\
                \end{eqnarray*}
                \columnbreak
                \begin{eqnarray*}
                    NR2 &=& \int_{0}^1\frac{2\,\alpha\,(1-x)^{\beta-1}\,x^{\alpha}}{\alpha+\beta}\cdot dx\\
                     &=& \frac{2\,\alpha}{\alpha+\beta}\int_0^1 (1-x)^{\beta-1}\cdot x^{\alpha}\cdot dx\\
                     &=& \frac{2\,\alpha}{\alpha+\beta}\int_0^1 (1-x)^{\beta-1}\cdot x^{\epsilon-1}\cdot dx\\
                     &=& \frac{2\,\alpha}{\alpha+\beta}\frac{\Gamma(\epsilon)\Gamma(\beta)}{\Gamma(\epsilon+\beta)}\\
                     &=& \frac{2\,\alpha}{\alpha+\beta}\frac{\Gamma(\alpha+1)\Gamma(\beta)}{\Gamma(\alpha+1+\beta)}\\
                     &=& \frac{2\,\alpha}{\alpha+\beta}\frac{\alpha\cdot \Gamma(\alpha)\Gamma(\beta)}{(\alpha+\beta)\Gamma(\alpha+\beta)}\\
                    NR2 &=& \frac{2\,\alpha^2}{(\alpha+\beta)^2}\frac{\Gamma(\alpha)\Gamma(\beta)}{\Gamma(\alpha+\beta)}\\
                \end{eqnarray*}
            \end{multicols}
            \begin{eqnarray*}
                \mathbb{V}(x) &=& \frac{1}{\beta(\alpha, \beta)}\cdot \left(\frac{\alpha(\alpha+1)\Gamma(\alpha)\Gamma(\beta)}{\Gamma(\alpha+\beta)(\alpha+1+\beta)(\alpha+\beta)} - \frac{2\,\alpha^2}{(\alpha+\beta)^2}\frac{\Gamma(\alpha)\Gamma(\beta)}{\Gamma(\alpha+\beta)} + \frac{\alpha^2}{(\alpha+\beta)^2} \frac{\Gamma(\alpha)\Gamma(\beta)}{\Gamma(\alpha+\beta)} \right)\\
                \mathbb{V}(x) &=& \frac{1}{\beta(\alpha, \beta)}\cdot \left(\frac{\alpha(\alpha+1)\Gamma(\alpha)\Gamma(\beta)(\alpha+\beta)- 2\,\alpha^2\Gamma(\alpha)\Gamma(\beta)(\alpha+\beta+1) + \alpha^2\Gamma(\alpha)\Gamma(\beta)(\alpha+\beta+1)}{\Gamma(\alpha+\beta)(\alpha+\beta+1)(\alpha+\beta)^2} \right)\\
                \mathbb{V}(x) &=& \frac{1}{\beta(\alpha, \beta)}\cdot \left(\frac{\Gamma(\alpha)\Gamma(\beta)\left(\alpha(\alpha+1)(\alpha+\beta)- \cancel{2}\,\alpha^2(\alpha+\beta+1) +\cancel{\alpha^2(\alpha+\beta+1)}\right)}{\Gamma(\alpha+\beta)(\alpha+\beta+1)(\alpha+\beta)^2} \right)\\
                \mathbb{V}(x) &=& \frac{1}{\beta(\alpha, \beta)}\cdot \left(\frac{\Gamma(\alpha)\Gamma(\beta)\left(\alpha(\alpha+1)(\alpha+\beta)- \alpha^2(\alpha+\beta+1)\right)}{\Gamma(\alpha+\beta)(\alpha+\beta+1)(\alpha+\beta)^2} \right)\\
                \mathbb{V}(x) &=& \frac{1}{\beta(\alpha, \beta)}\cdot \left(\frac{\Gamma(\alpha)\Gamma(\beta)\left(\cancel{\alpha^3}+\cancel{\alpha^2}+\cancel{\alpha^2\beta}+\alpha\beta - \cancel{\alpha^3}-\cancel{\alpha^2\beta}-\cancel{\alpha^2}\right)}{\Gamma(\alpha+\beta)(\alpha+\beta+1)(\alpha+\beta)^2} \right)\\
                \mathbb{V}(x) &=& \frac{1}{\beta(\alpha, \beta)}\cdot \left(\frac{\Gamma(\alpha)\Gamma(\beta)\alpha\beta}{\Gamma(\alpha+\beta)(\alpha+\beta+1)(\alpha+\beta)^2} \right)\fbox{Einsetzen der Beta-Fktn.}\\
                \mathbb{V}(x) &=& \frac{\cancel{\Gamma(\alpha+\beta)}}{\cancel{\Gamma(\alpha)\Gamma(\beta)}}\cdot \left(\frac{\cancel{\Gamma(\alpha)\Gamma(\beta)}\alpha\beta}{\cancel{\Gamma(\alpha+\beta)}(\alpha+\beta+1)(\alpha+\beta)^2} \right)\\
                \mathbb{V}(x) &=& \frac{\alpha\beta}{(\alpha+\beta+1)(\alpha+\beta)^2}\\
            \end{eqnarray*}
    \end{enumerate}
\end{Answer}
\end{uebsp}
