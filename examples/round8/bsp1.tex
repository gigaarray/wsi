\begin{uebsp}
\begin{Exercise}[label=ex:8.1]
Bestimmen Sie den Maximum-Likelihood-Schätzer für den Parameter $\lambda$ einer Poissonverteilung und zeigen Sie, dass er effizient ist.
\end{Exercise}
\begin{Answer}
\index{Maximum-Likelihood-Schätzer!Beispiel}
\index{Log-Likelihood-Funktion!Beispiel}
\index{Multiplikationssatz für Likelihood-Funktionen!Beispiel}
\index{Cramér-Rao, Satz!Beispiel}
\index{Fisher-Information!Beispiel}
\index{Effizienz!Beispiel}
\index{Erwartungstreue!Beispiel}
\index{Poissonverteilung!Beispiel}

\begin{uebsp_theory}
Die \reference{Likelihoodfunktion}{Definition}{def:likelihoodfunktion} in unserem Fall (diskrete Zufallsvariable) ist definiert, als 
\[L(X;\lambda)=P_\lambda(X)\;\;\text{ bzw. }\;\;\ln L(X;\lambda)=l(X;\lambda)\]
und da es sich um unabhängige Zufallsvariablen $X$ handelt, gilt der \reference{Multiplikationssatz}{Definition}{def:multiplikationssatz_likelihood}
\[L(\vec X;\lambda)=\prod_{i=1}^nf_\lambda(X)\]
bzw. für die Log-Likelihood:
\[\ln L(\vec X;\lambda)=l(\vec X;\lambda)=\sum_{i=1}^nl(X_i;\lambda)\]
die \reference{Poissonverteilung}{Kapitel}{sec:poissonverteilung} hat außerdem die folgende Wahrscheinlichkeitsfunktion:
\[
\mathbb P\left(X=k\right)=\frac{1}{k!}\lambda ^{k}e^{-\lambda }
\]
\end{uebsp_theory}
\begin{enumerate}[i)]
    \item \textbf{Likelihood-Funktion zusammenbauen:}
    Ein bisschen Umformen mit den \reference{Logarithmusregeln}{Anhang}{sec:logarithmus}:
        \begin{eqnarray*}
            l(\vec k;\lambda)&=&\sum_{i=1}^n\ln P(k_i) = \sum_{i=1}^n\ln\left(\frac{1}{k_i!}\lambda ^{k_i}e^{-\lambda }\right) = \sum_{i=1}^n\left(\ln\left(\frac{1}{k_i!}\right)+\ln\lambda ^{k_i}+\ln e^{-\lambda }\right)\\
            l(\vec k;\lambda)&=&\sum_{i=1}^n\left(\ln\left(k_i!\right)^{-1}+k_i\cdot \ln\lambda+-\lambda\cdot \underbrace{\ln e}_{=1}\right)\\
            l(\vec k;\lambda)&=&-n\cdot \lambda+\ln\lambda\cdot\sum_{i=1}^nk_i-\sum_{i=1}^n\ln(k_i!)
        \end{eqnarray*}
    \item \textbf{Ableiten der Likelihood-Funktion:}
        \begin{eqnarray*}
            \frac{\partial l(\vec x;\lambda)}{\partial\lambda}&=&\frac{\partial\left(-n\cdot \lambda+\ln\lambda\cdot\sum_{i=1}^nk_i-\sum_{i=1}^n\ln(k_i!)\right)}{\partial\lambda}\\
            \frac{\partial l(\vec x;\lambda)}{\partial\lambda}&=&\frac{\partial\left(-n\cdot \lambda\right)}{\partial\lambda}+\frac{\partial\left(\ln\lambda\cdot\sum_{i=1}^nk_i\right)}{\partial\lambda}-\frac{\partial\left(\sum_{i=1}^n\ln(k_i!)\right)}{\partial\lambda}\\
            \frac{\partial l(\vec x;\lambda)}{\partial\lambda}&=&-n+\frac{1}{\lambda}\cdot\sum_{i=1}^nk_i\\
        \end{eqnarray*}
    \item \textbf{Ableitung=0 für Maximum und auf $\lambda$ umstellen}
        \[\frac{\partial l(\vec x;\lambda)}{\partial\lambda}=0=-n+\frac{1}{\lambda}\cdot\sum_{i=1}^nk_i\Leftrightarrow n=\frac{1}{\lambda}\cdot\sum_{i=1}^nk_i\Leftrightarrow n\cdot \lambda=\sum_{i=1}^nk_i\]
        \[\hat\lambda=\frac{1}{n}\sum_{i=1}^nk_i=\overline x_n=\hat\lambda_n\]

    \item \textbf{Check, ob Erwartungstreu}
        \begin{uebsp_theory}
            Hier reicht es, zu zeigen, dass gilt: $\mathbb{E}_\lambda(\hat\lambda_n)=\lambda.$ 
            \reference{}{Definition}{def:eigenschaften_schaetzer}
        \end{uebsp_theory}
        \[\mathbb{E}(\hat\lambda)=\mathbb{E}(\overline k_n)=\mathbb{E}\left(\frac{1}{n}\sum_{i=1}^nk_i\right)=\frac{1}{n}\cdot\underbrace{\mathbb{E}\left(\sum_{i=1}^nk_i\right)}_{=n\cdot\lambda}=\frac{1}{\cancel n}\cdot \cancel n\cdot \lambda=\lambda\]
    \item \textbf{Check, ob Effizient}
        \begin{uebsp_theory}
            Nur ein Erwartungstreuer Schätzer kann auf effizient sein. Außerdem muss er unter allen erwartungstreuen Schätzern die kleinste Varianz besitzen.
            \reference{}{Definition}{def:eigenschaften_schaetzer}

            Außerdem muss mittels der \reference{Cramér-Rao-Schranke}{Satz}{satz:cramer_rao} überprüft werden, ob es sich um einen effizienten Schätzer handelt:
            \[\mathbb V(\hat\lambda_n)\ge {1\over I_n(\lambda)}={1\over nI(\lambda)}.\]

            Wobei die Fisher-Information wie folgt berechnet wird:
            \[I_n(\lambda)=-\mathbb E_\lambda\left({\partial^2\over\partial\lambda^2}l(k_1,\dots,k_n;\lambda)\right)\]
        \end{uebsp_theory}
        \begin{eqnarray*}
            \frac{\partial^2 l(\vec k;\lambda)}{\partial^2\lambda}&=&\frac{\partial\left(-n+\frac{1}{\lambda}\cdot\sum_{i=1}^nk_i\right)}{\partial\lambda}=-\frac{1}{\lambda^2}\cdot\sum_{i=1}^nk_i\\
        \end{eqnarray*}
        \[I_n(\lambda)=-\mathbb E_\lambda\left(-\frac{1}{\lambda^2}\cdot\underbrace{\sum_{i=1}^nk_i}_{=n\cdot\lambda}\right)=\frac{1}{\lambda^{\cancel 2}}n\cdot\cancel\lambda=\frac{n}{\lambda}\]
    \begin{eqnarray*}\mathbb V(\hat\lambda_n)&=&\mathbb V(\overline k_i)=\mathbb V(\frac{1}{n}\sum_{i=1}^nk_i)=\frac{1}{n^2}\mathbb V(\sum_{i=1}^nk_i)\framebox{... denn alles aus $\mathbb V$ wird quadriert.}\\
        \mathbb V(\hat\lambda_n) &=& \frac{1}{n^2}\sum_{i=1}^n\mathbb V(k_i)=\frac{1}{n^{\cancel2}}\cancel n\cdot\mathbb V(k)=\frac{\mathbb V(k)}{n}=\frac{\lambda}{n}\framebox{...$\mathbb V(k)=\lambda$ aus Tabelle}
    \end{eqnarray*}
    \[\mathbb V(\hat\lambda_n)=\frac{\lambda}{n}\ge \frac{1}{I_n(\lambda)}={\frac{\lambda}{n}}.\;\;\framebox{Es herrscht Gleichheit, somit effizient.}\]
\end{enumerate}
   
\end{Answer}
\end{uebsp}
