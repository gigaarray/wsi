\begin{uebsp}
\begin{Exercise}[label=ex:8.2]
Bestimmen Sie den Maximum-Likelihood-Schätzer für den Parameter einer Exponentialverteilung mit der Dichte \[f(x,\theta)=\frac{1}{\theta}e^{-x/\theta}\;\;[x\geq 0]\] und zeigen Sie, dass er effizient ist.
\end{Exercise}
\begin{Answer}
\index{Maximum-Likelihood-Schätzer!Beispiel}
\index{Log-Likelihood-Funktion!Beispiel}
\index{Multiplikationssatz für Likelihood-Funktionen!Beispiel}
\index{Cramér-Rao, Satz!Beispiel}
\index{Fisher-Information!Beispiel}
\index{Effizienz!Beispiel}
\index{Erwartungstreue!Beispiel}

\begin{uebsp_theory}
Die \reference{Likelihoodfunktion}{Definition}{def:likelihoodfunktion} in unserem Fall (stetige Zufallsvariable) ist definiert, als 
\[L(X;\theta)=f_\theta(X)\;\;\text{ bzw. }\;\;\ln L(X;\theta)=l(X;\theta)\]
und da es sich um unabhängige Zufallsvariablen $X$ handelt, gilt der \reference{Multiplikationssatz}{Definition}{def:multiplikationssatz_likelihood}
\[L(\vec X;\theta)=\prod_{i=1}^nf_\theta(X)\]
bzw. für die Log-Likelihood:
\[\ln L(\vec X;\theta)=l(\vec X;\theta)=\sum_{i=1}^nl(X_i;\theta)\]
\end{uebsp_theory}
\begin{enumerate}[i)]
    \item \textbf{Likelihood-Funktion zusammenbauen:}
    Ein bisschen Umformen mit den \reference{Logarithmusregeln}{Anhang}{sec:logarithmus}:
        \begin{eqnarray*}
            l(\vec x;\theta)&=&\sum_{i=1}^n\ln f(x_i) = \sum_{i=1}^n\ln\left(\frac{1}{\theta}e^{-x_i/\theta}\right)=\sum_{i=1}^n\left(\ln\frac{1}{\theta}-\frac{x_i}{\theta}\ln\left(e\right)\right)\\
            l(\vec x;\theta)&=&n\ln\frac{1}{\theta}+\sum_{i=1}^n\left(-\frac{x_i}{\theta}\right) = n\ln\theta^{-1} -\frac{1}{\theta}\sum_{i=1}^{n}x_i=-n\ln\theta -\frac{1}{\theta}\sum_{i=1}^{n}x_i
        \end{eqnarray*}
    \item \textbf{Ableiten der Likelihood-Funktion:}
        \begin{eqnarray*}
            \frac{\partial l(\vec x;\theta)}{\partial\theta}&=&\frac{\partial\left(-n\ln\theta -\frac{1}{\theta}\sum_{i=1}^{n}x_i\right)}{\partial\theta}=\frac{\partial\left(-n\ln\theta\right)}{\partial\theta} -\frac{\partial\left(\frac{1}{\theta}\sum_{i=1}^{n}x_i\right)}{\partial\theta}\\
            \frac{\partial l(\vec x;\theta)}{\partial\theta}&=&-\frac{n}{\theta} +\frac{1}{\theta^2}\sum_{i=1}^{n}x_i
        \end{eqnarray*}
    \item \textbf{Ableitung=0 für Maximum und auf $\theta$ umstellen}
        \[\frac{\partial l(\vec x;\theta)}{\partial\theta}=0=-\frac{n}{\theta} +\frac{1}{\theta^2}\sum_{i=1}^{n}x_i\;\Leftrightarrow\;\frac{n}{\cancel\theta}=\frac{1}{\theta^{\cancel{2}}}\sum_{i=1}^nx_i\;\Leftrightarrow\;n\cdot \theta=\sum_{i=1}^nx_i\]
        \[\hat\theta_n=\frac{1}{n}\sum_{i=1}^nx_i=\overline x_n\]
    \item \textbf{Check, ob Erwartungstreu}
        \begin{uebsp_theory}
            Hier reicht es, zu zeigen, dass gilt: $\mathbb{E}_\theta(\hat\theta_n)=\theta.$ 
            \reference{}{Definition}{def:eigenschaften_schaetzer}
        \end{uebsp_theory}
        \[\mathbb{E}(\hat\theta_n)=\mathbb{E}(\overline x_n)=\mathbb{E}\left(\frac{1}{n}\sum_{i=1}^nx_i\right)=\frac{1}{n}\cdot\underbrace{\mathbb{E}\left(\sum_{i=1}^nx_i\right)}_{=n\cdot\theta}=\frac{1}{\cancel n}\cdot \cancel n\cdot \theta=\theta\]
    \item \textbf{Check, ob Effizient}
        \begin{uebsp_theory}
            Nur ein Erwartungstreuer Schätzer kann auf effizient sein. Außerdem muss er unter allen erwartungstreuen Schätzern die kleinste Varianz besitzen.
            \reference{}{Definition}{def:eigenschaften_schaetzer}

            Außerdem muss mittels der \reference{Cramér-Rao-Schranke}{Satz}{satz:cramer_rao} überprüft werden, ob es sich um einen effizienten Schätzer handelt:
            \[\mathbb V(\hat\theta_n)\ge {1\over I_n(\theta)}={1\over nI(\theta)}.\]

            Wobei die Fisher-Information wie folgt berechnet wird:
            \[I_n(\theta)=-\mathbb E_\theta\left({\partial^2\over\partial\theta^2}l(x_1,\dots,x_n;\theta)\right)\]
        \end{uebsp_theory}
        \begin{eqnarray*}
            \frac{\partial^2 l(\vec x;\theta)}{\partial^2\theta}&=&\frac{\partial\left(-\frac{n}{\theta} +\frac{1}{\theta^2}\sum_{i=1}^{n}x_i\right)}{\partial\theta}=\frac{n}{\theta^2}-\frac{2}{\theta^3}\cdot\sum_{i=1}^nx_i\\
        \end{eqnarray*}
        \[I_n(\theta)=-\mathbb E_\theta\left(\frac{n}{\theta^2}-\frac{2}{\theta^3}\cdot\sum_{i=1}^nx_i\right)=-\frac{n}{\theta^2}+\frac{2}{\theta^3}\cdot n\theta=\frac{2n}{\theta^2}-\frac{n}{\theta^2}=\frac{n}{\theta^2}\]
    \begin{eqnarray*}\mathbb V(\hat\theta_n)&=&\mathbb V(\overline x_i)=\mathbb V(\frac{1}{n}\sum_{i=1}^nx_i)=\frac{1}{n^2}\mathbb V(\sum_{i=1}^nx_i)\framebox{$\mathbb V\Rightarrow x^2$ \reference{}{Definition}{def:varianz_eigenschaften}.}\\
        \mathbb V(\hat\theta_n) &=& \frac{1}{n^2}\sum_{i=1}^n\mathbb V(x_i)=\frac{1}{n^{\cancel2}}\cancel n\cdot\mathbb V(x)=\frac{\mathbb V(x)}{n}=\frac{\theta^2}{n}\framebox{...$\mathbb V(x)=\theta^2$ aus Tabelle}
    \end{eqnarray*}
    \[\mathbb E(x)=\int_{-\infty}^\infty xf_X(x)dx=\int_{0}^\infty\frac{x}{\theta}e^{-x/\theta}dx=\frac{1}{\theta}\int_{0}^\infty x e^{-x/\theta}=\frac{1}{\theta}\left.\theta\left(-e^{-\frac{x}{\theta}}\right)(\theta+x)\right|_{x=0}^\infty\]
    \[\mathbb E(x)=\left.\left(-e^{-\frac{x}{\theta}}\right)(\theta+x)\right|_{x=0}^\infty=\underbrace{\left(-e^{-\frac{\infty}{\theta}}\right)(\theta+\infty)}_{\rightarrow0}+\left(e^{-\frac{0}{\theta}}\right)(\theta+0)=\theta\]
    \[\mathbb V(x)=\mathbb{E}(x^2)-\mathbb{E}^2(x)=-\theta^2\]
    \[\mathbb V(\hat\theta_n)=\frac{\theta^2}{n}\ge \frac{1}{I_n(\theta)}={\frac{\theta^2}{n}}.\;\;\framebox{Es herrscht Gleichheit, somit effizient.}\]

\end{enumerate}
   
\end{Answer}
\end{uebsp}
