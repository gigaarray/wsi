\begin{uebsp}
\begin{Exercise}[label=ex:4.1]
Zwei Spieler A und B mit Kapital $a$ und $b$ spielen folgendes Spiel: In jeder Runde setzt jeder Spieler einen Einsatz 1. Dann wird eine Münze geworfen, und A gewinnt, wenn sie ''Kopf'' zeigt, sonst gewinnt B. Das Spiel ist zu Ende, wenn ein Spieler kein Kapital mehr hat. Überlegen Sie, dass $X_t$, das Kapital von A zum Zeitpunkt $t$, eine Markovkette bildet, bestimmen Sie die Übergangsmatrix und die Klassen und ihre Perioden.
\end{Exercise}
\begin{Answer}

\begin{uebsp_theory}
    Bei Markovketten 1.Ordnung hängt die Zukunft nur von der Gegenwart (dem aktuellen Zustand) ab.
\end{uebsp_theory}

\begin{description}
    \item [Die Markovkette]:\\
        \begin{tikzpicture}[->,>=stealth',shorten >=1pt,auto,node distance=0.95cm,
                    semithick]
  \tikzstyle{every state}=[fill=blue!30,draw=none,text=white,minimum size=0.35cm]

  \node[state]              (0)                                         {};
  \node[state]              (1)      [right of=0]                       {};
  \node[state]              (2)      [right of=1]                       {};
  \node[state, fill=none]   (3)      [right of=2]                       {};
  \node[state, fill=none]   (a-2)    [right of=3, node distance=0.8cm]  {};
  \node[state]              (a-1)    [right of=a-2]                     {};
  \node[state]              (a)      [right of=a-1]                     {};
  \node[state]              (a+1)    [right of=a]                       {};
  \node[state, fill=none]   (a+2)    [right of=a+1]                     {};
  \node[state, fill=none]   (a+b-3)  [right of=a+2, node distance=0.8cm]{};
  \node[state]              (a+b-2)  [right of=a+b-3]                   {};
  \node[state]              (a+b-1)  [right of=a+b-2]                   {};
  \node[state]              (a+b)    [right of=a+b-1]                   {};

  \node (desc0) [below of=0,node distance=1.6cm, align=center] {Zustand 0\\Absorbierend};
  \node (desca+b) [below of=a+b,node distance=1.6cm, align=center] {Zustand a+b\\Absorbierend};
  \node (desca) [below of=a,node distance=1.6cm, align=center] {Zustand a\\Start};

  \path (0) edge [loop above]   node {$1$} (0)
        (1) edge                node [above]{$\frac{1}{2}$} (0)
            edge [bend right]   node [below]{$\frac{1}{2}$} (2)
        (2) edge [bend right]   node [above]{$\frac{1}{2}$} (1)
            edge [bend right]   node [below]{$\frac{1}{2}$} (3)
        (3) edge [bend right]   node [above]{$\frac{1}{2}$} (2);

  \path (a+b)   edge [loop above]   node {$1$} (a+b)
        (a+b-1) edge                node [below]{$\frac{1}{2}$} (a+b)
                edge [bend right]   node [above]{$\frac{1}{2}$} (a+b-2)
        (a+b-2) edge [bend right]   node [below]{$\frac{1}{2}$} (a+b-1)
                edge [bend right]   node [above]{$\frac{1}{2}$} (a+b-3)
        (a+b-3) edge [bend right]   node [below]{$\frac{1}{2}$} (a+b-2);

  \path (a-2)   edge [bend right]   node [below]{$\frac{1}{2}$} (a-1)
        (a-1)   edge [bend right]   node [above]{$\frac{1}{2}$} (a-2)
                edge [bend right]   node [below]{$\frac{1}{2}$} (a)
        (a)     edge [bend right]   node [above]{$\frac{1}{2}$} (a-1)
                edge [bend right]   node [below]{$\frac{1}{2}$} (a+1)
        (a+1)   edge [bend right]   node [above]{$\frac{1}{2}$} (a)
                edge [bend right]   node [below]{$\frac{1}{2}$} (a+2)
        (a+2)   edge [bend right]   node [above]{$\frac{1}{2}$} (a+1);

  \path [draw, dotted, -] (3) -- (a-2)
                          (a+2) -- (a+b-3);

  \path [draw, dashed, ->] (desc0) -- (0);
  \path [draw, dashed, ->] (desca+b) -- (a+b);
  \path [draw, dashed, ->] (desca) -- (a);

\end{tikzpicture}

    \item [Die Übergangsmatrix]:
\[P=\left(\begin{array}{cccccc}
    1 & 0 & 0 & 0 & \cdot\cdot\cdot & 0\\
    1/2 & 0 & 1/2 & 0 & \cdot\cdot\cdot & 0\\
    0 & 1/2 & 0 & 1/2 & \cdot\cdot\cdot & 0\\
    0 & 0 & 1/2 & 0 & \cdot\cdot\cdot & 0\\
    \vdots & \vdots & \vdots & \vdots& \ddots & \vdots\\
    0 & 0 & 0 & 0 & \cdot\cdot\cdot & 1\\
\end{array}\right)\text{\parbox{0.5\linewidth}{Wobei man hier links oben bzw. rechts unten gut die absorbierenden Zustände sieht.}}\]

\begin{uebsp_theory}
    Der Zustand $j$ heißt \reference{Nachfolger}{Definition}{def:nachfolger} von $i$ ($i\rightarrow j$), wenn es ein $t\ge 0$ gibt, sodass $p_{ij}(t)>0$.\index{Nachfolger!Beispiel}\\
    Wenn sowohl $i\rightarrow j$ als auch $j\rightarrow i$ gilt, dann heißen $i$ und $j$ verbunden oder kommunizierend.
\end{uebsp_theory}

\begin{uebsp_theory}
    Die \reference{Übergangswahrscheinlichkeit}{Section}{sec:uebergangswahrscheinlichkeit} ist definiert als \[p_{ij}\mathbb P(X_{n+1}=j|X_n=i)\] und gibt die Wahrscheinlichkeit des übergangs vom Zustand $i$ zum Zustand $j$ an. \index{"Ubergangswahrscheinlichkeit@Übergangswahrscheinlichkeit!Beispiel}
\end{uebsp_theory}

\begin{uebsp_theory}
    Eine Eigenschaft heißt \reference{Klasseneigenschaft}{Definition}{def:klasseneigenschaft}, wenn sie entweder für alle Zustände einer Klasse oder für keinen gilt.
    \index{Klasseneigenschaft!Beispiel}
\end{uebsp_theory}

\begin{uebsp_theory}
    Ein \reference{Absorbierender Zustand}{Klasseneigenschaften}{sec:klasseneigenschaften} ist als eigene Klasse definiert.
    \index{absorbierender Zustand!Beispiel}
\end{uebsp_theory}

    \item [Klassen]:
        \begin{itemize}
            \item $C_1={0}$
            \item $C_2={1,2,3,...,a+b-1}$
            \item $C_3={a+b}$ 
        \end{itemize}

\begin{uebsp_theory}
    Die \reference{Periode}{Definition}{def:periode} eines Zustandes ist:
    \index{Periode!Beispiel}
    \[d(i)=\mathrm{ggT}\{t\ge 0:p_{ii}(t)>0\}.\]
\end{uebsp_theory}

    \item [Perioden]:
        \begin{itemize}
            \item $d(0)=ggT\{1\}=1$
            \item $d({1,2,3,...,a+b-1})=ggT\{2,4,6,8,...\}=2$ (denn man benötigt mindestens $2,4,6,8,...$ Schritte, um vom Zustand $i$ wieder zum Zustand $i$ zurückzukehren.)
            \item $d(a+b)=ggT\{1\}=1$
        \end{itemize}
\end{description}
\end{Answer}
\end{uebsp}
