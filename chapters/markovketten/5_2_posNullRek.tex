% This file was converted to LaTeX by Writer2LaTeX ver. 1.0.2
% see http://writer2latex.sourceforge.net for more info
%\documentclass[a4paper]{article}
%\usepackage[ascii]{inputenc}
%\usepackage[T1]{fontenc}
%\usepackage[english,english]{babel}
%\usepackage{amsmath}
%\usepackage{amssymb,amsfonts,textcomp}
%\usepackage{color}
%\usepackage{array}
%\usepackage{hhline}
%\usepackage{hyperref}
%\hypersetup{pdftex, colorlinks=true, linkcolor=blue, citecolor=blue, filecolor=blue, urlcolor=blue, pdftitle=, pdfauthor=, pdfsubject=, pdfkeywords=}
%\usepackage[pdftex]{graphicx}
% Page layout (geometry)
%\setlength\voffset{-1in}
%\setlength\hoffset{-1in}
%\setlength\topmargin{2cm}
%\setlength\oddsidemargin{2cm}
%\setlength\textheight{25.7cm}
%\setlength\textwidth{17.001cm}
%\setlength\footskip{0.0cm}
%\setlength\headheight{0cm}
%\setlength\headsep{0cm}
% Footnote rule
%\setlength{\skip\footins}{0.119cm}
%\renewcommand\footnoterule{\vspace*{-0.018cm}\setlength\leftskip{0pt}\setlength\rightskip{0pt plus 1fil}\noindent\textcolor{black}{\rule{0.25\columnwidth}{0.018cm}}\vspace*{0.101cm}}
% Pages styles
%\makeatletter
%\newcommand\ps@Standard{
%  \renewcommand\@oddhead{}
%  \renewcommand\@evenhead{}
%  \renewcommand\@oddfoot{}
%  \renewcommand\@evenfoot{}
%  \renewcommand\thepage{\arabic{page}}
%}
%\makeatother
%\pagestyle{Standard}
%\title{}
%\author{}
%\date{2014-02-25T10:01:05.420000000}
%\begin{document}
\begin{uebsp}
{\centering\selectlanguage{english}\bfseries
Altes Beispiel 5.2
\par}


%\bigskip

{\selectlanguage{english}
Nun wollen wir noch feststellen ob die Markovkette aus 5.1 positiv- oder
nullrekurrent ist.}
\index{Nullrekurrent!Beispiel}
\index{positiv Rekurrent!Beispiel}\\

\begin{center}\textbf{Lösung zu Übung 5.2}\end{center}

{\selectlanguage{english}
Wir wissen:}


%\bigskip

{\selectlanguage{english}
Wenn eine station\"are Verteilung f\"ur die Markovkette existiert, dann
ist die Markovkette positiv rekurrent, ansonsten nullrekurrent.}


%\bigskip

{\selectlanguage{english}
Also versuchen wir eine station\"are Verteilung zu berechnen, d.h.: wir
berechnen:}


%\bigskip

{\selectlanguage{english}
 $\pi _{j}=\sum _{i}\pi _{i}p_{\mathit{ij}}$ was gelichbedeutend ist
mit:  $\underline{{\pi }}\cdot P=\underline{{\pi }}$ (vgl.
Matrizenmultiplikation)}


%\bigskip

{\selectlanguage{english}
wobei  $\pi _{j}\text{ (bzw. }\pi _{i}\text{)}$ die Eintr\"age des
Wahrscheinlichkeitsvektors  $\underline{{\pi }}$ sind, welcher als
Zeilenvektor aufgefasst wird und .  $\pi _{j}$ Beschreibt also die
Wahrscheinlichkeit daf\"ur, dass sich die Kette in in Zustand j
aufh\"alt. Im Falle einer station\"aren Verteilung ist diese ja auch
vom Zeitpunkt unabh\"angig.}


%\bigskip

{\selectlanguage{english}
Wir schauen wiedermal als erstes auf unsere  $p_{\mathit{ij}}$ denn von
denen wissen wir, dass sie oft 0 sind.}


%\bigskip

{\selectlanguage{english}
So bleibt uns  $\pi _{j}=\pi _{i-2}\cdot p+\pi _{i+1}(1-p)$ f\"ur p
eingesetzt ergibt das:}


%\bigskip

\begin{equation*}
\pi _{j}=\frac{1}{3}\pi _{i-2}\cdot +{\frac{2}{3}}\cdot \pi _{i+1}
\end{equation*}
\begin{equation*}
2\pi _{i+1}-3\pi _{i}+\pi _{i-2}=0
\end{equation*}

%\bigskip


%\bigskip

{\selectlanguage{english}
So erhalten wir mit  $2z^{3}-3z^{2}+1=0$:  $z_{1}=1\text{ durch
probieren}$ und dann f\"ur  $2z^{2}-z-1=0$ :  $z_{2}=1\text{ und
}z_{3}=-{\frac{1}{2}}$ }


%\bigskip

{\selectlanguage{english}
Das wiederum f\"uhrt uns zur L\"osung:}

\begin{equation*}
\pi _{i}=(C_{1}+C_{2}i)z_{1,2}^{i}+C_{3}z_{3}^{i}
\end{equation*}
\begin{equation*}
\pi _{i}=C_{1}+iC_{2}+(-{\frac{1}{2}})^{i}C_{3}
\end{equation*}

%\bigskip

{\selectlanguage{english}
Unsere Zustandsmenge ist prinzipiell \"uber  $\mathbb{Z}$
identifizierbar. Das bedeutet aber auch, dass es Zust\"ande mit einer
''ID'' i gibt die gegen unendlich geht.
Bilden wir also f\"ur i den Grenzwert gegen  $\pm \infty $ wo es
schlie{\ss}lich auch Zust\"ande gibt, so stellen wir fest:}


%\bigskip

\begin{equation*}
\pi _{i}=C_{1}+\underbrace{iC_{2}}_{\rightarrow \pm \infty
}+\underbrace{(-{\frac{1}{2}})^{i}C_{3}}_{\rightarrow 0}
\end{equation*}
{\selectlanguage{english}
Da aber f\"ur die station\"are Verteilung aber gelten muss:  $\sum
_{i}\pi _{i}=1$ ,ergibt sich ein Widerspruch, d.h.: es existiert keine
station\"are Verteilung. Somit ist die Markovkette nullrekurrent.}


%\bigskip


%\bigskip


%\bigskip
%\end{document}
\end{uebsp}
