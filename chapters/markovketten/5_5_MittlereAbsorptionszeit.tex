% This file was converted to LaTeX by Writer2LaTeX ver. 1.0.2
% see http://writer2latex.sourceforge.net for more info
%\documentclass[a4paper]{article}
%\usepackage[ascii]{inputenc}
%\usepackage[T1]{fontenc}
%\usepackage[english,ngerman]{babel}
%\usepackage{amsmath}
%\usepackage{amssymb,amsfonts,textcomp}
%\usepackage{color}
%\usepackage{array}
%\usepackage{hhline}
%\usepackage{hyperref}
%\hypersetup{pdftex, colorlinks=true, linkcolor=blue, citecolor=blue, filecolor=blue, urlcolor=blue, pdftitle=, pdfauthor=, pdfsubject=, pdfkeywords=}
% Page layout (geometry)
%\setlength\voffset{-1in}
%\setlength\hoffset{-1in}
%\setlength\topmargin{2cm}
%\setlength\oddsidemargin{2cm}
%\setlength\textheight{25.7cm}
%\setlength\textwidth{17.001cm}
%\setlength\footskip{0.0cm}
%\setlength\headheight{0cm}
%\setlength\headsep{0cm}
% Footnote rule
%\setlength{\skip\footins}{0.119cm}
%\renewcommand\footnoterule{\vspace*{-0.018cm}\setlength\leftskip{0pt}\setlength\rightskip{0pt plus 1fil}\noindent\textcolor{black}{\rule{0.25\columnwidth}{0.018cm}}\vspace*{0.101cm}}
% Pages styles
%\makeatletter
%\newcommand\ps@Standard{
%  \renewcommand\@oddhead{}
%  \renewcommand\@evenhead{}
%  \renewcommand\@oddfoot{}
%  \renewcommand\@evenfoot{}
%  \renewcommand\thepage{\arabic{page}}
%}
%\makeatother
%\pagestyle{Standard}
%\title{}
%\author{}
%\date{2014-02-25T10:01:15.251000000}
%\begin{document}
\begin{uebsp}

{\centering\bfseries
Altes Beispiel 5.5
\par}


%\bigskip
\index{mittlere Absorptionszeit!Beispiel}
Um die mittlere Zeit bis zur Absorption einmal praktisch zu berechnen
kehren wir zur\"uck zu unserem Beispiel 4.1. \ Unsere absorbierenden
Zust\"ande sind hier: \{0\}: {\quotedblbase}Spieler A hat kein Kapital
mehr und verloren{\textquotedblleft} \ und \{a+b\}:
{\quotedblbase}Spieler A hat alles und gewonnen{\textquotedblleft}.\\


\begin{center}\textbf{Lösung zu Übung 5.5}\end{center}

%\bigskip

Somit wissen wir bereits:  $m_{0}=0$ und  $m_{a+b}=0$ 


%\bigskip

F\"ur unsere anderen Zust\"ande gilt: 

\begin{equation*}
m_{i}=1+\sum _{j}p_{\mathit{ij}}m_{j}
\end{equation*}

%\bigskip

Wenn das Summenzeichen im ersten Moment auch etwas abschreckend wirkt,
wird sich gleich herausstellen, dass es halb so schlimm ist, da ohnehin
die meisten  $p_{\mathit{ij}}=0$ sind. \ Denn von unserem Zustand i
gelangen wir nur zu den beiden Nachbarzust\"anden i-1 und i+1 und das
jeweils mit einer Wahrscheinlichkeit von  $\frac{1}{2}$ . D.h.: 
$m_{i}=1+\frac{1}{2}m_{i-1}+\frac{1}{2}m_{i+1}$


%\bigskip

Sehen wir uns nun an wie die Folge der  $m_{i}$ verl\"auft. 


%\bigskip


%\bigskip

\begin{equation*}
m_{1}=1+\frac{1}{2}m_{0}+\frac{1}{2}m_{2}=1+0+\frac{1}{2}m_{2}=1+\frac{1}{2}m_{2}
\end{equation*}

%\bigskip

\begin{equation*}
\rightarrow m_{2}=2m_{1}-2=2(m_{1}-1)
\end{equation*}

%\bigskip

 $m_{2}=1+\frac{1}{2}m_{1}+\frac{1}{2}m_{3}$  daraus folgt: 
$m_{3}=2m_{2}-2-m_{1}$ setzen wir nun f\"ur  $m_{2}$ den oben
erhaltenen Ausdruck ein, erhalten wir:


%\bigskip


\bigskip

\begin{equation*}
m_{3}=2(2(m_{1}-1))-2-m_{1}=4m_{1}-4-2-m_{1}=3m_{1}-6=3(m_{1}-2)
\end{equation*}
Wir k\"onnen also folgende Vermutung aufstellen:


%\bigskip


%\bigskip

\begin{equation*}
m_{i}=i(m_{1}-(i-1))=i(m_{1}+1-i)
\end{equation*}
Nun w\"are es noch sch\"on zu erfahren wie gro{\ss}  $m_{1}$ konkret
ist. Dies k\"onnen wir auch berechnen, da wir \ wissen, dass
$m_{a+b}=0$.


%\bigskip


%\bigskip

\begin{equation*}
m_{a+b}=(a+b)(m_{1}+1-(a+b))=(a+b)(m_{1}+1-a-b)=0
\end{equation*}
Wir k\"onnen davon ausgehen, dass  $(a+b)$ gr\"o{\ss}er 0 ist. Damit der
Ausdruck oben also 0 wird, muss  $(m_{1}+1-a-b)=0$ sein.


%\bigskip

So erhalten wir: 


%\bigskip

\begin{equation*}
m_{1}=a+b-1
\end{equation*}

\bigskip

Damit k\"onnen wir die mittlere Absorptionszeit berechnen f\"ur den
Fall, dass wir in Zustand a (so viel Kapital hat Spieler A zu beginn)
starten.


\bigskip

\begin{equation*}
m_{a}=a(m_{1}-(a-1))=a(a+b-1-a+1)=a(b)=\mathit{ab}
\end{equation*}
\begin{equation*}
m_{b}=b(m_{1}-(b-1))=b(a+b-1-b+1)=b(a)=\mathit{ab}
\end{equation*}

%\bigskip

%\end{document}
\end{uebsp}
