\begin{uebsp}
{\centering\selectlanguage{english}\bfseries
Beispiel 4.3
\par}


%\bigskip

{\selectlanguage{english}
Das Beispiel 4.1 k\"onnen wir heranziehen um einmal praktisch die
Berechnung der Absorptionswahrscheinlichkeit durchzuf\"uhren. Spieler A
gewinnt, wenn er \{a+b\} Einheiten bestitzt. Das Spiel ist dann zu
Ende, d.h.: der Zustand \{a+b\} ist ein absorbierender Zustand.}


%\bigskip

{\selectlanguage{english}
 $a_{i_{0}}=1$  d.h.:  $a_{a+b}=1$ }


%\bigskip

\begin{equation*}
a_{i}=\sum _{j}p_{\mathit{ij}}a_{j}
\end{equation*}

%\bigskip

{\selectlanguage{english}
Von dieser Summe bleibt nicht sehr viel \"ubrig, denn die einzigen
Zust\"ande j die von \  $i$ aus erreichbar sind (und f\"ur die damit 
$p_{\mathit{ij}}\neq 0$ ist) sind die Nachbarzust\"ande, welche jeweils
mit Wahrscheinlichkeit  $\frac{1}{2}$ erreicht werden. \ Wir erhalten
also:}


%\bigskip

\begin{equation*}
a_{i}=p_{i,i-1}a_{i-1}+p_{i,i+1}a_{i+1}=\frac{1}{2}a_{i-1}+\frac{1}{2}a_{i+1}
\end{equation*}

%\bigskip

{\selectlanguage{english}
Es gibt hier 2 L\"osungsans\"atze: im Allgemeinen wird man hier wohl
nicht drum-herum kommen, \ eine Differenzengleichnung nach allen Regeln
der Kunst zu l\"osen, deshalb wird hier sp\"ater auch gezeigt wie das
funktioniert. Man kann sich in diesem speziellen Fall, die Aufgabe
etwas erleichtert in dem man die Gleichung auf folgende Weise umformt:}


%\bigskip

\begin{equation*}
a_{i+1}-a_{i}=a_{i}-a_{i-1}=c\text{ ( konstant )}
\end{equation*}
{\selectlanguage{english}
So kann man erkennen, dass die Differenz zwischen zwei benachbarten a
immer konstant ist. D.h.: aber auch, dass gilt:}


%\bigskip

\begin{equation*}
a_{i}=a_{0}+i\cdot c=i\cdot c
\end{equation*}

%\bigskip

{\selectlanguage{english}
Die Konstante c k\"onnen wir noch auf folgende Weise berechnen:}


%\bigskip

\begin{equation*}
a_{a+b}=1=(a+b)\cdot c\text{   }\rightarrow \text{   }c=\frac{1}{(a+b)}
\end{equation*}
{\selectlanguage{english}
Und wir erhalten:}


%\bigskip

\begin{equation*}
a_{a}=\frac{a}{(a+b)}
\end{equation*}
{\selectlanguage{english}
Nun sei noch gezeigt, wie man durch l\"osen der Differenzengleichung auf
die L\"osung gekommen w\"are. Wir bringen unsere Gleichung zuerst auf
die Form:}


%\bigskip

\begin{equation*}
a_{i+1}-2a_{i}-a_{i-1}=0
\end{equation*}

%\bigskip


%\bigskip

{\selectlanguage{english}
Dies ist eine lineare Differenzengleichung zweiter Ordnung mit
konstanten Koeffizienten. Eine solche ist im Allgemeinen von folgender
Form:}


%\bigskip

{\selectlanguage{english}
 $x_{n+2}+\mathit{ax}_{n+1}+\mathit{bx}_{n}=s_{n}$  mit  $n\geqslant 0$
}


%\bigskip

{\selectlanguage{english}
Wobei a und b konstante Koeffizienten (mit  $b\neq 0$) sind und der
St\"orfunktion  $s_{n}$ m\"oglicherweise von n abh\"angt. Ist 
$s_{n}=0$ f\"ur alle n, so spricht von man von einer homogenen,
andernfalls von einer inhomogenen Gleichung.}


%\bigskip

{\selectlanguage{english}
Wir rufen uns in Erinnerung wie man so etwas l\"ost:}


%\bigskip

{\selectlanguage{english}
1. Bestimmen der allgemeinen L\"osung der homogenen Gleichung 
$x_{n}^{(h)}$ }

{\selectlanguage{english}
2. Bestimmen einer partikul\"aren L\"osung  $x_{n}^{(p)}$ }

{\selectlanguage{english}
3. Ermitteln der L\"osungsgesamtheit gem\"a{\ss} 
$x_{n}=x_{n}^{(h)}+x_{n}^{(p)}$ }


%\bigskip

{\selectlanguage{english}
Nun denn, beginnen wir mit der L\"osung der homogenen Gleichung:}


%\bigskip

\begin{equation*}
x_{n+2}+\mathit{ax}_{n+1}+\mathit{bx}_{n}=0
\end{equation*}

%\bigskip

{\selectlanguage{english}
Man w\"ahlt den Ansatz  $x_{n}^{(h)}=\lambda ^{n}$  und erh\"alt:}


%\bigskip

\begin{equation*}
\lambda ^{n+2}+a\lambda ^{n+1}+b\lambda ^{n}=0
\end{equation*}

%\bigskip

{\selectlanguage{english}
hier kann man noch durch  $\lambda ^{n}$ k\"urzen und bekommt:}


%\bigskip

\begin{equation*}
\lambda ^{2}+a\lambda +b=0
\end{equation*}

%\bigskip

{\selectlanguage{english}
Diese Quadratische Gleichnung wird auch ''Charkteristische
Gleichung'' genannt. Wichtig zu wissen ist, wie die
L\"osungen  $\lambda _{1}$ und \  $\lambda _{2}$ zu interpretieren sind
bzw. was sie f\"ur unser  $x_{n}^{(h)}$ bedeuten: }


%\bigskip


%\bigskip


%\bigskip

\begin{equation*}
x_{n}^{(h)}=\left\{\begin{matrix}C_{1}\lambda _{1}^{n}+C_{2}\lambda
_{2}^{n}\text{  falls }\lambda _{1}\neq \lambda _{2}\hfill\null
\\r^{n}(C_{1}\cos (n\phi )+C_{2}\cos (n\phi ))\text{   falls }\lambda
_{1,2}=r(\cos (\phi )\pm i\sin (\phi ))\text{ konjugiert
komplex}\hfill\null \\(C_{1}+\mathit{nC}_{2})\lambda _{1}^{n}\text{  
falls}\lambda _{1}=\lambda _{2}\text{reell}\hfill\null
\end{matrix}\right.
\end{equation*}

%\bigskip


%\bigskip

{\selectlanguage{english}
mit  $C_{1,}C_{2}\in \mathbb{R}$ }


%\bigskip

{\selectlanguage{english}
Im Schritt 2 ist dann im Allgemeinen noch eine partikul\"are L\"osung zu
finden, indem man unter zu Hilfenahme der folgenden Tabelle ein Ansatz
f\"ur  $x_{n}^{(p)}$ macht und entsprechend umformt.}


%\bigskip



\begin{center}
\includegraphics[width=17cm,height=4.62cm]{chapters/markovketten/a43Absorptionswahrscheinlichkeit-img1.pdf}
\end{center}
{\selectlanguage{english}
In diesem Beispiel ist der ganze Bl\"odsinn nicht notwendig, da wir,
wenn wir geschickt umstellen auch anders auf unsere L\"osung kommen.
Der Vollst\"andigkeit halber wollen wir das nun aber durch exerzieren
-- es ist ja auch nicht so viel arbeit, da sowieso nur einen homogene
Differenzengleichung vorliegt.}


%\bigskip

\begin{equation*}
a_{i+1}-2a_{i}-a_{i-1}=0
\end{equation*}

%\bigskip


%\bigskip

\begin{equation*}
\lambda ^{2}-2\lambda -1=0
\end{equation*}

%\bigskip

{\selectlanguage{english}
Zur Erinnerung: die quadratische L\"osungsformel f\"ur 
$\mathit{ax}^{2}+\mathit{bx}+c=0$ lautet:}


%\bigskip

\begin{equation*}
x_{1,2}=\frac{-b\pm \sqrt{b^{2}-4\mathit{ac}}}{2a}
\end{equation*}
{\selectlanguage{english}
und f\"ur  $x^{2}+\mathit{px}+q=0$:}


%\bigskip

\begin{equation*}
x_{1,2}=-{\frac{p}{2}}\pm \sqrt{(\frac{p}{2})^{2}-q}
\end{equation*}
\begin{equation*}
\lambda _{1,2}=\frac{2}{2}\pm \sqrt{(\frac{2}{2})^{2}-1}=1\pm 0=1
\end{equation*}

%\bigskip


%\bigskip

{\selectlanguage{english}
Damit ist unsere L\"osung }

\begin{equation*}
x_{n}^{(h)}=a_{n}=(C_{1}+\mathit{nC}_{2})\lambda
_{1}^{n}=(C_{1}+\mathit{nC}_{2})(1)^{n}=(C_{1}+\mathit{nC}_{2})
\end{equation*}
{\selectlanguage{english}
Wir wissen, dass \  $a_{a+b}=1$ und ebenso wissen wir, dass  $a_{0}=0$
(dies ist der Fall wenn B und A kein Kapital mehr hat -- auch dann ist
das Spiel zu Ende, die Wahrscheinlichkeit zu gewinnen ist dann aber 0,
da A ja bereits verloren hat!) Daraus k\"onnen wir uns unsere 
$C_{1,}C_{2}\in \mathbb{R}$ berechnen.}


%\bigskip

{\selectlanguage{english}
 $a_{0}=0=(C_{1}+0C_{2})=(C_{1}+0)=0$ Daher:  $C_{1}=0$ }


%\bigskip

\begin{equation*}
a_{a+b}=(C_{1}+(a+b)C_{2})=(0+(a+b)C_{2})=1
\end{equation*}
{\selectlanguage{english}
Letzendlich erhalten wir so:}

\begin{equation*}
C_{2}=\frac{1}{a+b}
\end{equation*}

%\bigskip

{\selectlanguage{english}
Somit lautet die Formel f\"ur unsere Absorptionswahrscheinlichkeiten:}


%\bigskip

\begin{equation*}
a_{n}=n\frac{1}{a+b}
\end{equation*}
{\selectlanguage{english}
Laut Angabe beginnen wir bei dem Zustand a -- d.h. Spieler A hat zu
Beginn Kaptial a.}


%\bigskip

{\selectlanguage{english}
Die Absorptionswahrscheinlichkeit  $a_{a}$ ist also:}


%\bigskip

\begin{equation*}
a_{a}=\frac{a}{a+b}
\end{equation*}

%\bigskip


%\bigskip
%\end{document}
\end{uebsp}
